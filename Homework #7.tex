\documentclass{article}

\setlength{\headsep}{0.75 in}
\setlength{\parindent}{0 in}
\setlength{\parskip}{0.1 in}

%=====================================================
% Add PACKAGES Here (You typically would not need to):
%=====================================================

\usepackage{xcolor}
\usepackage[margin=1in]{geometry}
\usepackage{amsmath,amsthm}
\usepackage{fancyhdr}
\usepackage{enumitem}
\usepackage{graphicx}
\usepackage{amsmath, amssymb}  % Include the amsmath and amssymb packages for mathematical symbols

%=====================================================
% Ignore This Part (But Do NOT Delete It:)
%=====================================================

\theoremstyle{definition}
\newtheorem{problem}{Problem}
\newtheorem*{fun}{Fun with Algorithms}
\newtheorem*{challenge}{Challenge Yourself}
\def\fline{\rule{0.75\linewidth}{0.5pt}}
\newcommand{\finishline}{\begin{center}\fline\end{center}}
\newtheorem*{solution*}{Solution}
\newenvironment{solution}{\begin{solution*}}{{\finishline} \end{solution*}}
\newcommand{\grade}[1]{\hfill{\textbf{($\mathbf{#1}$ points)}}}
\newcommand{\thisdate}{April 11, 2024}
\newcommand{\thissemester}{\textbf{Rutgers: Spring 2024}}
\newcommand{\thiscourse}{ECE 509: Convex Optimization} 
\newcommand{\thishomework}{Number} 
\newcommand{\thisname}{Name} 
\newcommand{\thisextension}{Yes/No} 

\headheight 40pt              
\headsep 10pt
\renewcommand{\headrulewidth}{0pt}
\pagestyle{fancy}

\newcommand{\thisheading}{
   \noindent
   \begin{center}
   \framebox{
      \vbox{\vspace{2mm}
    \hbox to 6.28in { \textbf{\thiscourse \hfill \thissemester} }
       \vspace{4mm}
       \hbox to 6.28in { {\Large \hfill Homework \#\thishomework \hfill} }
       \vspace{2mm}
         \hbox to 6.28in { { \hfill \thisdate  \hfill} }
       \vspace{2mm}
       \hbox to 6.28in { \emph{Name: \thisname \hfill Extension: \thisextension}}
      \vspace{2mm}}
      }
   \end{center}
   \bigskip
}

%=====================================================
% Some useful MACROS (you can define your own in the same exact way also)
%=====================================================


\newcommand{\ceil}[1]{{\left\lceil{#1}\right\rceil}}
\newcommand{\floor}[1]{{\left\lfloor{#1}\right\rfloor}}
\newcommand{\prob}[1]{\Pr\paren{#1}}
\newcommand{\expect}[1]{\Exp\bracket{#1}}
\newcommand{\var}[1]{\textnormal{Var}\bracket{#1}}
\newcommand{\set}[1]{\ensuremath{\left\{ #1 \right\}}}
\newcommand{\poly}{\mbox{\rm poly}}


%=====================================================
% Fill Out This Part With Your Own Information:
%=====================================================


\renewcommand{\thishomework}{7} %Homework number
\renewcommand{\thisname}{Ravi Raghavan} % Enter your name here
\renewcommand{\thisextension}{No} % Pick only one of the two options accordingly

\begin{document}

\thisheading
\vspace{-0.75cm}


%=====================================================
% LaTeX Tip: You can erase this part from here.... 
%=====================================================		

\finishline

%=====================================================
% LaTeX Tip: ... to here
%=====================================================	


\bigskip

\begin{problem}
Problem 2.16: \newline 
    Show that if $S_1$ and $S_2$ are convex sets in $\mathbb{R}^{m + n}$, then so is their partial sum \newline 
    $S = \{(x, y_1 + y_2) : x \in \mathbb{R}^m, y_1, y_2 \in \mathbb{R}^n, (x, y_1) \in S_1, (x, y_2) \in S_2 \}$

    \begin{solution}

    Let's say that we have two points in $S$, namely $(x_1, y_{11} + y_{12})$ and $(x_2, y_{21} + y_{22})$. To prove that $S$ is convex, we need to show that $\theta (x_1, y_{11} + y_{12}) + (1 - \theta) (x_2, y_{21} + y_{22})$ is in $S$.  

    Based on the definition of $S$, we can see the following: 
    \begin{itemize}
        \item $(x_1, y_{11}) \in S_1$
        \item $(x_1, y_{12}) \in S_2$
        \item $(x_2, y_{21}) \in S_1$
        \item $(x_2, y_{22}) \in S_2$
    \end{itemize}

    Since $S_1$ and $S_2$ are convex, based on the definition of convex sets, we can see that:
    \begin{itemize}
        \item $\theta (x_1, y_{11}) + (1 - \theta) (x_2, y_{21}) \in S_1$

        $ (\theta x_1 + (1 - \theta) x_2, \theta y_{11} + (1 - \theta) y_{21} ) \in S_1$
        \item $\theta (x_1, y_{12})  + (1 - \theta)(x_2, y_{22}) \in S_2$

        $ (\theta x_1 + (1 - \theta) x_2, \theta y_{12} + (1 - \theta) y_{22} ) \in S_1$
    \end{itemize}

    By definition of Set $S$, we can see that: \newline 

    $ (\theta x_1 + (1 - \theta) x_2, \theta y_{11} + (1 - \theta) y_{21} + \theta y_{12} + (1 - \theta) y_{22} ) \in S$ \newline 

        $ (\theta x_1 + (1 - \theta) x_2, \theta (y_{11} + y_{12}) + (1 - \theta) (y_{21} + y_{22}) ) \in S$ \newline 

$\theta (x_1, y_{11} + y_{12}) + (1 - \theta) (x_2, y_{21} + y_{22})$ is in $S$.  
    
        
    \end{solution}
\end{problem}


\end{document}





