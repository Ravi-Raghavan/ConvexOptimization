\documentclass{article}

\setlength{\headsep}{0.75 in}
\setlength{\parindent}{0 in}
\setlength{\parskip}{0.1 in}

%=====================================================
% Add PACKAGES Here (You typically would not need to):
%=====================================================

\usepackage{xcolor}
\usepackage[margin=1in]{geometry}
\usepackage{amsmath,amsthm}
\usepackage{fancyhdr}
\usepackage{enumitem}
\usepackage{graphicx}
\usepackage{amsmath, amssymb}  % Include the amsmath and amssymb packages for mathematical symbols

%=====================================================
% Ignore This Part (But Do NOT Delete It:)
%=====================================================

\theoremstyle{definition}
\newtheorem{problem}{Problem}
\newtheorem*{fun}{Fun with Algorithms}
\newtheorem*{challenge}{Challenge Yourself}
\def\fline{\rule{0.75\linewidth}{0.5pt}}
\newcommand{\finishline}{\begin{center}\fline\end{center}}
\newtheorem*{solution*}{Solution}
\newenvironment{solution}{\begin{solution*}}{{\finishline} \end{solution*}}
\newcommand{\grade}[1]{\hfill{\textbf{($\mathbf{#1}$ points)}}}
\newcommand{\thisdate}{April 25, 2024}
\newcommand{\thissemester}{\textbf{Rutgers: Spring 2024}}
\newcommand{\thiscourse}{ECE 509: Convex Optimization} 
\newcommand{\thishomework}{Number} 
\newcommand{\thisname}{Name} 
\newcommand{\thisextension}{Yes/No} 

\headheight 40pt              
\headsep 10pt
\renewcommand{\headrulewidth}{0pt}
\pagestyle{fancy}

\newcommand{\thisheading}{
   \noindent
   \begin{center}
   \framebox{
      \vbox{\vspace{2mm}
    \hbox to 6.28in { \textbf{\thiscourse \hfill \thissemester} }
       \vspace{4mm}
       \hbox to 6.28in { {\Large \hfill Homework \#\thishomework \hfill} }
       \vspace{2mm}
         \hbox to 6.28in { { \hfill \thisdate  \hfill} }
       \vspace{2mm}
       \hbox to 6.28in { \emph{Name: \thisname \hfill Extension: \thisextension}}
      \vspace{2mm}}
      }
   \end{center}
   \bigskip
}

%=====================================================
% Some useful MACROS (you can define your own in the same exact way also)
%=====================================================


\newcommand{\ceil}[1]{{\left\lceil{#1}\right\rceil}}
\newcommand{\floor}[1]{{\left\lfloor{#1}\right\rfloor}}
\newcommand{\prob}[1]{\Pr\paren{#1}}
\newcommand{\expect}[1]{\Exp\bracket{#1}}
\newcommand{\var}[1]{\textnormal{Var}\bracket{#1}}
\newcommand{\set}[1]{\ensuremath{\left\{ #1 \right\}}}
\newcommand{\poly}{\mbox{\rm poly}}


%=====================================================
% Fill Out This Part With Your Own Information:
%=====================================================


\renewcommand{\thishomework}{8} %Homework number
\renewcommand{\thisname}{Ravi Raghavan} % Enter your name here
\renewcommand{\thisextension}{No} % Pick only one of the two options accordingly

\begin{document}

\thisheading
\vspace{-0.75cm}


%=====================================================
% LaTeX Tip: You can erase this part from here.... 
%=====================================================		

\finishline

%=====================================================
% LaTeX Tip: ... to here
%=====================================================	


\bigskip
\begin{problem}
\textit{Composition with an affine function.} Show that the following function $f: \mathbb{R}^n \rightarrow \mathbb{R}$ are convex
\begin{itemize}
    \item[(a)] $f(x) = ||Ax - b||$, where $A \in \mathbb{R}^{m x n}$, $b \in \mathbb{R}^m$, and $||.||$ is a norm on $\mathbb{R}^m$
    \begin{solution}
    Let us have $x, y \in dom f$. Our goal is, for $\theta \in [0, 1]$ to prove that $f(\theta x + (1 - \theta) y ) \leq \theta f(x) + (1 - \theta) f(y)$

    $f(\theta x + (1 - \theta) y ) = ||A(\theta x + (1 - \theta) y) - b|| = ||\theta 
 Ax + (1 - \theta) Ay - b||$

The triangle inequality theorem(i.e. $||x|| + ||y|| \geq ||x + y||$) also states that: \newline 
$\theta f(x) + (1 - \theta) f(y) = \theta ||Ax - b|| + (1 - \theta) ||Ay - b|| =  ||\theta(Ax - b)|| + ||(1 - \theta)(Ay - b)|| \geq ||\theta (Ax - b) + (1 - \theta) (Ay - b)|| = ||\theta Ax - \theta b + (1 - \theta) Ay -  (1 - \theta)b|| = ||\theta Ax  + (1 - \theta) Ay - b|| = f(\theta x + (1 - \theta) y )$ \newline 

Another way of doing this proof is just by seeing that $f(x)$ is a composition of a norm and an affine function. Norms are convex. Hence, since we have a composition with an affine mapping, we have a convex function! 
 
    \end{solution} 

    \item[(b)] $f(x) = - (\det (A_0 + x_1 A_1 + \dots + x_n A_n))^{\frac{1}{m}}$, on $\{x | A_0 + x_1 A_1 + \dots + x_n A_n \succ 0\}$ where $A_i \in \mathbb{S}^m$

    \begin{solution}
        Let $\lambda_{ij}$ represent the $j$th Eigenvalue of $A_i$. \newline 

        Since $A_i \in \mathbb{S}^m$, the eigenvalues of $A_0 + x_1 A_1 + \dots + x_n A_n$ are just the sum of the individual eigenvalues of $A_0, x_1 A_1, \dots x_n A_n$

        We already know that the determinant of a Symmetric Matrix is the product of its eigenvalues. Hence, we can re-express $f(x)$ as follows: \newline 

        $f(x) = - ((\lambda_{01} + x_1 \lambda_{11} + \dots + x_n \lambda_{n1}) \dots (\lambda_{0m} + x_1 \lambda_{1m} + \dots + x_n \lambda_{nm}))^{\frac{1}{m}}$

        Since $A_0 + x_1 A_1 + \dots + x_n A_n \succ 0$, we know that $(\lambda_{0j} + x_1 \lambda_{1j} + \dots + x_n \lambda_{nj}) > 0$

Hence, we can clearly rewrite our problem: \newline 

        Let us denote $P = \begin{bmatrix}
    \lambda_{11} & \cdots & \lambda_{n1} \\
    \lambda_{12} & \cdots & \lambda_{n2} \\
    \vdots & \ddots & \vdots \\
    \lambda_{1m} & \cdots & \lambda_{nm}
\end{bmatrix}$

Let $Q = \begin{bmatrix}
    \lambda_{01} \\
    \lambda_{02} \\
    \vdots \\
    \lambda_{0m}
\end{bmatrix}$

Let $h(x) = - (\prod_{i=1}^{m} x_i)^{\frac{1}{m}}$ where $x \in \mathbb{R}^m_{++}$ \newline 

Let $g(x) = Px + Q$

$f(x) = h(g(x))$ \newline 

Since geometric means are concave as we proved in class where the domain is $\mathbb{R}^m_{++}$, $h(x)$ is convex. Furthermore, when the domain is $\mathbb{R}^m_{++}$, it is clear that $h(x)$ is non-increasing in each argument. 

$g(x)$ is an affine transformation is, thus, convex and concave. 
        
Hence, we have a case where $h(x)$ is convex and non-increasing and $g(x)$ is concave. This means that $f(x) = h(g(x))$ is Convex. 

As shown, we have shown that $f$ is a Composition of a Convex Function and an Affine Mapping, hence telling us that $f$ is Convex
    \end{solution}
\end{itemize}
\end{problem}

\begin{problem}
    {Pointwise maximum and supremum.} Show that the following function $f: \mathbb{R}^n \rightarrow \mathbb{R}$ are convex

    \begin{itemize}
        \item[(a)] $f(x) = \max_{i = 1, \dots, k} || A^{(i)}x - b^{(i)}||$, where $A^{(i)} \in \mathbb{R}^{m x n}$, $b^{(i)} \in \mathbb{R}^m$, and $||.||$ is a norm on $\mathbb{R}^m$

        \begin{solution}
            This function is just a pointwise maximum of a bunch of convex functions(this was proved earlier), which are convex. Hence, $f$ is convex. 
        \end{solution}

        \item[(b)] $f(x) = \sum_{i=1}^{r} |x|_{[i]}$ on $\mathbb{R}^n$, where $|x|$ denotes the vector with $|x|_i = |x_i|$ (i.e. $|x|$ is the absolute value $x$, component wise), and $|x|_{[i]}$ is the $i$th largest component of $|x|$.  In other words,  $|x|_{[1]}, |x|_{[2]}, \dots , |x|_{[n]}$ are the absolute values of the components of $x$, sorted in nonincreasing order.

        \begin{solution}
            We can re-express $f(x)$: \newline 

            $f(x) = \max_{i_1, i_2, \dots, i_r} \{|x_{i_1}| + |x_{i_2}| + \dots + |x_{i_r}|\}$ where $i_k \in [1, n] \enspace \forall k \in [1, r]$ \newline 

            It is clear that $f(x)$ is the pointwise maximum of $\binom{n}{r}$ convex functions. 

            
        \end{solution}
    \end{itemize}
\end{problem}


\begin{problem}
\textit{Composition rules.} Show that the following functions are convex.

\begin{itemize}
    \item[(a)] $f(x) = -\log(-\log(\sum_{i=1}^{m} e^{a_i^Tx + b_i}))$ on $dom f = \{ x | \sum_{i=1}^{m} e^{a_i^Tx + b_i} < 1\}$. You can use the fact that $\log(\sum_{i=1}^{m} e^{y_i})$ is convex. 

    \begin{solution}
        We can re-express $f(x)$ as follows: \newline 

        $f(x) = h(g(Ax + b))$ where $h(x) = - \log(x)$ and $g(x) = -\log(\sum_{i=1}^{m} e^{y_i})$ and $A$ is a matrix such that each $a_i$ is a row of $A$ and $b$ is a vector such that each $b_i$ is a row of $b$. 

        Let's start by first analyzing $g(Ax + b)$. We know that $g$ is concave. Hence, $g(Ax + b)$ is concave. 

        Since $h(x) = - \log(x)$, $h(x)$ is convex. 

        Since we have $h$ is convex, $\tilde{h}$ is nonincreasing and $g(Ax + b)$ is concave, we have that $f$ is convex. 
    \end{solution}

    \item[(b)] $f(x, t) = -\log(t^p - ||x||^p_p)$ where $p > 1$ and $dom f = \{(x, t) | t > ||x||_p \}$. You can use the fact that $\frac{||x||^p_p}{u^{p - 1}}$ is convex in $(x, u)$ for $u > 0$

    \begin{solution}
        Let's start rewriting $f(x, t)$ \newline 

        $f(x, t) = -\log(t^{p - 1}(t - \frac{||x||^p_p}{t^{p - 1}}))$ \newline 

        $f(x, t) = -\log(t^{p - 1}) - \log((t - \frac{||x||^p_p}{t^{p - 1}}))$

        $f(x, t) = -(p - 1)\log(t) - \log((t - \frac{||x||^p_p}{t^{p - 1}}))$

        The function $-(p - 1)\log(t)$ is convex since $p > 1$(i.e. $p - 1 > 0$) and $-\log(t)$ is convex. \newline 

        Let's look at $- \log((t - \frac{||x||^p_p}{t^{p - 1}}))$. This function is a composition of a non-increasing convex function(i.e. $-\log(x)$) and a concave function(i.e. $(t - \frac{||x||^p_p}{t^{p - 1}})$). 

        Let's briefly explain why $(t - \frac{||x||^p_p}{t^{p - 1}})$ is concave. $\frac{||x||^p_p}{t^{p - 1}}$ is convex. Hence, $-\frac{||x||^p_p}{t^{p - 1}}$ is concave. $t$ can be viewed as an affine map. Hence, $t$ is convex and concave. The sum of 2 concave functions is concave. Hence, $(t + (- \frac{||x||^p_p}{t^{p - 1}})) = (t - \frac{||x||^p_p}{t^{p - 1}})$ is Concave. 

        Hence, $- \log((t - \frac{||x||^p_p}{t^{p - 1}}))$ is convex. 

        Since $-(p - 1)\log(t)$ and $- \log((t - \frac{||x||^p_p}{t^{p - 1}}))$ are both convex, $f(x, t) = -(p - 1)\log(t) - \log((t - \frac{||x||^p_p}{t^{p - 1}}))$ is convex as well since it is the sum of $-(p - 1)\log(t)$ and $- \log((t - \frac{||x||^p_p}{t^{p - 1}}))$
    \end{solution}
\end{itemize}
    
\end{problem}


\end{document}





