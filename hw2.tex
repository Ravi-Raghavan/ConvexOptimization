\documentclass{article}

\setlength{\headsep}{0.75 in}
\setlength{\parindent}{0 in}
\setlength{\parskip}{0.1 in}

%=====================================================
% Add PACKAGES Here (You typically would not need to):
%=====================================================

\usepackage{xcolor}
\usepackage[margin=1in]{geometry}
\usepackage{amsmath,amsthm}
\usepackage{fancyhdr}
\usepackage{enumitem}
\usepackage{graphicx}
\usepackage{amsmath, amssymb}  % Include the amsmath and amssymb packages for mathematical symbols

%=====================================================
% Ignore This Part (But Do NOT Delete It:)
%=====================================================

\theoremstyle{definition}
\newtheorem{problem}{Problem}
\newtheorem*{fun}{Fun with Algorithms}
\newtheorem*{challenge}{Challenge Yourself}
\def\fline{\rule{0.75\linewidth}{0.5pt}}
\newcommand{\finishline}{\begin{center}\fline\end{center}}
\newtheorem*{solution*}{Solution}
\newenvironment{solution}{\begin{solution*}}{{\finishline} \end{solution*}}
\newcommand{\grade}[1]{\hfill{\textbf{($\mathbf{#1}$ points)}}}
\newcommand{\thisdate}{February 13, 2024}
\newcommand{\thissemester}{\textbf{Rutgers: Spring 2024}}
\newcommand{\thiscourse}{ECE 509: Convex Optimization} 
\newcommand{\thishomework}{Number} 
\newcommand{\thisname}{Name} 
\newcommand{\thisextension}{Yes/No} 

\headheight 40pt              
\headsep 10pt
\renewcommand{\headrulewidth}{0pt}
\pagestyle{fancy}

\newcommand{\thisheading}{
   \noindent
   \begin{center}
   \framebox{
      \vbox{\vspace{2mm}
    \hbox to 6.28in { \textbf{\thiscourse \hfill \thissemester} }
       \vspace{4mm}
       \hbox to 6.28in { {\Large \hfill Homework \#\thishomework \hfill} }
       \vspace{2mm}
         \hbox to 6.28in { { \hfill \thisdate  \hfill} }
       \vspace{2mm}
       \hbox to 6.28in { \emph{Name: \thisname \hfill Extension: \thisextension}}
      \vspace{2mm}}
      }
   \end{center}
   \bigskip
}

%=====================================================
% Some useful MACROS (you can define your own in the same exact way also)
%=====================================================


\newcommand{\ceil}[1]{{\left\lceil{#1}\right\rceil}}
\newcommand{\floor}[1]{{\left\lfloor{#1}\right\rfloor}}
\newcommand{\prob}[1]{\Pr\paren{#1}}
\newcommand{\expect}[1]{\Exp\bracket{#1}}
\newcommand{\var}[1]{\textnormal{Var}\bracket{#1}}
\newcommand{\set}[1]{\ensuremath{\left\{ #1 \right\}}}
\newcommand{\poly}{\mbox{\rm poly}}


%=====================================================
% Fill Out This Part With Your Own Information:
%=====================================================


\renewcommand{\thishomework}{2} %Homework number
\renewcommand{\thisname}{Ravi Raghavan} % Enter your name here
\renewcommand{\thisextension}{No} % Pick only one of the two options accordingly

\begin{document}

\thisheading
\vspace{-0.75cm}


%=====================================================
% LaTeX Tip: You can erase this part from here.... 
%=====================================================		

\finishline

%=====================================================
% LaTeX Tip: ... to here
%=====================================================	


\bigskip

\begin{problem} Let $f : \mathbb{R}^n \rightarrow \mathbb{R}$ be a continuous function with dom $f$ being closed. Prove that the function is closed. 

\begin{solution}
Let's analyze any given sublevel set of the function $f$. Given any constant $\alpha \in \mathbb{R}$, a sublevel set is defined as \newline 

$S = \{x \in dom f : f(x) \leq \alpha \}$

Our goal is to prove that any such sub-level set is closed. 

Let's say we have a convergent sequence $H = (x_n)_{n \in \mathbb{N}}$ which lies in the sublevel set $S$ and we know that this sequence $H$ converges to c. 

Since $H$ lies within $S$ and $S$ is a subset of the domain of $f$, we know that $H$ lies within the domain of $f$. 

Since the domain of $f$ is a closed set and $H$ lies within the domain of $f$, we know that $c$ lies within the domain of $f$. 

Since the function is continuous, $H$ lies within the domain of $f$, and $c$ lies within the domain of $f$, we know that the sequence $(f(x_n))_{n \in \mathbb{N}}$ converges to $f(c)$. 

Since $H$ lies within $S$, we know that $(f(x_n))_{n \in \mathbb{N}}$ is always $\leq \alpha$. \newline 


Limit Property: \newline 
Given two sequences $(a_n)_{n \in \mathbb{N}}$ and $(b_n)_{n \in \mathbb{N}}$, if $a_n \leq b_n$ for all $n$, then  $\lim_{n\to\infty} a_n \leq \lim_{n\to\infty} b_n$. 

Usage of Limit Property in Proof: \newline 
If we consider $a_n$ to be $f(x_n)$ and consider $b_n$ to be a constant sequence where the value is always $\alpha$, we can see the following: \newline 

$f(x_n) \leq \alpha \rightarrow \lim_{n\to\infty} f(x_n) \leq \lim_{n\to\infty} \alpha$ \newline 

$f(x_n) \leq \alpha \rightarrow f(c) \leq \alpha$ \newline 

Since we can say that $f(c) \leq \alpha$ and we know that $c$ is in the domain of $f$, we know that $c$ is within $S$. \newline 


Hence, since $H$ lies within $S$ and $H$ converges to $c$ and $c$ is in $S$, we have proved that $S$ is a closed set. \newline 


By proving that any such sub-level set is closed, we have showed that the function $f$ is closed! \newline 
\end{solution}
    

%=====================================================
% LaTeX Tip: Feel free to erase the Example/Hint parts when writing your solution if they are in the way (when doing so, make sure you do not erase the
% other parts of LaTeX commands such as \end{enumerate} or \end{problem} -- however, please NEVER erase the problem statements. 
%=====================================================		


\end{problem}

\begin{problem}
Let $f : \mathbb{R}^n \rightarrow \mathbb{R}$ be a continuous function with dom $f$ being open. Prove that the function is closed if and only if for every sequence $x_i \in dom f$ such that $\lim_{i} x_i = x \in bd \enspace dom f$, we have $\lim_{i\to\infty} f(x_i) = \infty$

\begin{solution} Proof 

Step 1: "if" direction \newline 
We know that the function  $f$ is continuous and the domain $f$ is open. Let's analyze any given sublevel set of the function $f$. Given any constant $\alpha \in \mathbb{R}$, a sublevel set is defined as $S = \{x \in dom f : f(x) \leq \alpha \}$

Our goal is to prove that any such sub-level set is closed when for every sequence $x_i \in dom f$ such that $\lim_{i} x_i = x \in bd \enspace dom f$, we have $\lim_{i\to\infty} f(x_i) = \infty$. Let's say we have a convergent sequence $H = (x_n)_{n \in \mathbb{N}}$ which lies in the sublevel set $S$ and we know that this sequence $H$ converges to c. 

Since $H$ lies within $S$ and $S$ is a subset of the domain of $f$, we know that $H$ lies within the domain of $f$. 

Since $H$ lies within the domain of $f$ and the domain of $f$ is an OPEN set, there are usually a few cases for where $c$ can be. We know that limit points of convergent sequences in an open set either lie within the interior of the set or on the boundary of the set. Hence, the first case is that $c$ is on the Boundary of the domain. The second case is that $c$ is within the domain. 

However, if we look at the first case, this fact jumps out at us: for every sequence $x_i \in dom f$ such that $\lim_{i} x_i = x \in bd \enspace dom f$, we have $\lim_{i\to\infty} f(x_i) = \infty$. This would mean that $H$ CANNOT possibly be within $S$ since we need $f(x_i) \leq \alpha$ to ALWAYS be true. This would then violate our assumption at the beginning that $H$ lies in the sublevel set $S$. Hence, we know that only the second case must be true which means that $c$ is in the domain of $f$.   \newline 

Case: $c$ lies within the domain of $f$ \newline 
Since the function is continuous, $H$ lies within the domain of $f$, and $c$ lies within the domain of $f$, we know that the sequence $(f(x_n))_{n \in \mathbb{N}}$ converges to $f(c)$. 

Since $H$ lies within $S$, we know that $(f(x_n))_{n \in \mathbb{N}}$ is always $\leq \alpha$. \newline 


Limit Property: \newline 
Given two sequences $(a_n)_{n \in \mathbb{N}}$ and $(b_n)_{n \in \mathbb{N}}$, if $a_n \leq b_n$ for all $n$, then  $\lim_{n\to\infty} a_n \leq \lim_{n\to\infty} b_n$. 

Usage of Limit Property in Proof: \newline 
If we consider $a_n$ to be $f(x_n)$ and consider $b_n$ to be a constant sequence where the value is always $\alpha$, we can see the following: \newline 

$f(x_n) \leq \alpha \rightarrow \lim_{n\to\infty} f(x_n) \leq \lim_{n\to\infty} \alpha$ \newline 

$f(x_n) \leq \alpha \rightarrow f(c) \leq \alpha$ \newline 

Since we can say that $f(c) \leq \alpha$ and we know that $c$ is in the domain of $f$, we know that $c$ is within $S$. \newline 


Hence, since $H$ lies within $S$ and $H$ converges to $c$ and $c$ is in $S$, we have proved that $S$ is a closed set. \newline 


By proving that any such sub-level set is closed, we have showed that the function $f$ is closed! \newline 




Step 2: "only if" direction \newline 
We know that the function $f$ is continuous and closed and the domain $f$ is open. Let's analyze any given sublevel set of the function $f$. Given any constant $\alpha \in \mathbb{R}$, a sublevel set is defined as $S = \{x \in dom f : f(x) \leq \alpha \}$. Due to the fact that the function $f$ is closed, we know that any such sublevel set is CLOSED. The definition of a closed set is that the limit point of every CONVERGENT sequence is within that set. 

Let us have a sequence. We can describe this sequence as $x_i \in dom f$ such that $\lim_{i} x_i = x \in bd \enspace dom f$. We know that the sequence is contained within the domain of $f$, $f$ is continuous along its domain, and $f$ is closed. Hence, based on the general definition of limits, we can see that there are two cases for what we can observe for $\lim_{i\to\infty} f(x_i)$. This limit can either have a finite value or an infinite value. To prove that this limit must have an infinite value, I will show that it CANNOT have a finite value. \newline 

Case: The Limit is a finite value(i.e. $\lim_{i\to\infty} f(x_i) = c$ where $c \in \mathbb{R}$) . \newline 

Since our sequence is defined as $x_i \in dom f$ such that $\lim_{i} x_i = x \in bd \enspace dom f$, we know that $f(x_i)$ is defined at every point along the sequence.  Hence, in this case, the sequence $f(x_i)$ would be a CONVERGENT SEQUENCE. \newline 

Given the case where $\lim_{i\to\infty} f(x_i) = c$ where $c \in \mathbb{R}$, the sequence($x_i \in dom f$ such that $\lim_{i} x_i = x \in bd \enspace dom f$), based on the definition of a sublevel set, would be in $S$. \newline 

The reason is that convergent sequences are bounded. So, if we take $M$ to be the "maximum" value that can be attained in the sequence $f(x_i)$, we can say that the entire sequence($x_i \in dom f$ such that $\lim_{i} x_i = x \in bd \enspace dom f$) will be in the sublevel set $S_M = \{x \in dom f : f(x) \leq M \}$. \newline 

In this case, the limit point of the sequence($x_i \in dom f$ such that $\lim_{i} x_i = x \in bd \enspace dom f$) would NOT be in $S$ since the domain of $f$ is open. This would mean that $S$ is an open set which CONTRADICTS the fact that we know that $S$ HAS to be closed. \newline 

We have shown that $\lim_{i\to\infty} f(x_i) = \infty$\newline 

This completes the proof! 

\end{solution}
\end{problem}


\begin{problem}
Let $C \in \mathbb{R}^n$ be a convex set, with $x_1, ..., x_k \in C$, and let $\theta_1, ...., \theta_k \in \mathbb{R}$ satisfy $\theta_i \geq 0$, $\theta_1 + ... + \theta_k = 1$. Show that $\theta_1x_1 + ... + \theta_kx_k \in C$. (The definition of convexity is that this holds for k = 2; you must show it for arbitrary k.)

\begin{solution}
    Proof by Induction \newline 

Base Case: (k = 2) \newline 
This is true by definition of convex set \newline 

Inductive Step: \newline
Let's say this holds true for $k$. We must show that this holds true for $k + 1$ \newline 

We know that $\theta_1x_1 + ... + \theta_kx_k \in C$ given that $\theta_i \geq 0$, $\theta_1, ...., \theta_k \in \mathbb{R}$, and $\theta_1 + ... + \theta_k = 1$ \newline 

Let's introduce $\theta_{k + 1} \in \mathbb{R}$ such that $\theta_{i} \geq 0$ and $\theta_i \in \mathbb{R}$ , for $1 \leq i \leq k + 1$, and let's also say that $\theta_1 + ... + \theta_k + \theta_{k + 1} = 1$. It is evident to see that $\theta_{i} \leq 1$, for $1 \leq i \leq k + 1$, must be true as well. \newline

We want to prove that $\theta_1x_1 + ... + \theta_kx_k + \theta_{k + 1}x_{k + 1} \in C$ \newline 

Let's do some re-writing of mathematical expressions! \newline 

$\theta_1x_1 + ... + \theta_kx_k + \theta_{k + 1}x_{k + 1}$ \newline 
$(\theta_1x_1 + ... + \theta_kx_k) + \theta_{k + 1}x_{k + 1}$ \newline 

$(1 - \theta_{k + 1})(\frac{\theta_1}{1 - \theta_{k + 1}}x_1 + ... + \frac{\theta_k}{1 - \theta_{k + 1}}x_k) + \theta_{k + 1}x_{k + 1}$ \newline 


Since $\theta_1 + ... + \theta_k + \theta_{k + 1} = 1$, we know that $\theta_1 + ... + \theta_k = 1 - \theta_{k + 1}$. Hence, $\frac{\theta_1}{1 - \theta_{k + 1}} + ... + \frac{\theta_k}{1 - \theta_{k + 1}} = 1$ \newline 

Since $\theta_{k + 1} \in \mathbb{R}$ and $\theta_{k + 1} \geq 0$, we can see that $1 - \theta_{k + 1} \leq 1$ and $1 - \theta_{k + 1} \in \mathbb{R}$. Since $\theta_{k + 1} \leq 1$, $1 - \theta_{k + 1} \geq 0$ \newline 

Since $\theta_{i} \geq 0$, $1 - \theta_{k + 1} \geq 0$, and $\theta_i \in \mathbb{R}$ , for $1 \leq i \leq k + 1$, we can see that $\frac{\theta_i}{1 - \theta_{k + 1}} \geq 0$ and $\frac{\theta_i}{1 - \theta_{k + 1}} \in \mathbb{R}$

Since, $\frac{\theta_1}{1 - \theta_{k + 1}} + ... + \frac{\theta_k}{1 - \theta_{k + 1}} = 1$, $\frac{\theta_i}{1 - \theta_{k + 1}} \geq 0$, and $\frac{\theta_i}{1 - \theta_{k + 1}} \in \mathbb{R}$ for $1 \leq i \leq k + 1$, we can say that $\frac{\theta_1}{1 - \theta_{k + 1}}x_1 + ... + \frac{\theta_k}{1 - \theta_{k + 1}}x_k \in C$ \newline 


Now, let's represent $\frac{\theta_1}{1 - \theta_{k + 1}}x_1 + ... + \frac{\theta_k}{1 - \theta_{k + 1}}x_k = x_p$ where, as we have already proven, $x_p \in C$ \newline 

Our goal is to now prove that $(1 - \theta_{k + 1})(x_p) + \theta_{k + 1}x_{k + 1}$. However, based on the base case(k = 2) which is true by the definition of convexity, we know that this must be true! \newline 


This completes the proof that $\theta_1x_1 + ... + \theta_kx_k + \theta_{k + 1}x_{k + 1} \in C$

\end{solution}
\end{problem}

\begin{problem}
    Show that a set is convex if and only if its intersection with any line is convex. Show that a set is affine if and only if its intersection with any line is affine.

    \begin{solution} Proof \newline 
        Step 1: Show that a set is convex if and only if its intersection with any line is convex 

        \begin{itemize}
            \item  Part 1: The "If" Part \newline 
        We know that the intersection of the set and any line is convex. Our goal is to show that the set itself is convex. \newline 

        Let us start with a set $C$. Let us also denote an arbitrary line as $L$. Let us denote two distinct points $m$ and $n$ that are in $C \cap L$. Since we know that $C \cap L$ is convex, we can state the following: \newline 

        $\alpha m + (1 - \alpha)n \in C \cap L$ where $\alpha \in [0, 1]$ \newline 

        This leads us to the following: \newline 

        $\alpha m + (1 - \alpha)n \in C$ where $\alpha \in [0, 1]$ \newline 
        $\alpha m + (1 - \alpha)n \in L$ where $\alpha \in [0, 1]$ \newline 

        Hence, by definition, $C$ is convex as well! 
        

            \item Part 2: The "Only If" Part \newline
        We know that the set is convex. Our goal is to prove that the intersection of the set and any line is convex as well. \newline 

        Let us start with a convex set $C$. Let us also denote an arbitrary line as $L$. \newline 

        Let us denote two distinct points $m$ and $n$ that are in $C \cap L$. For $C \cap L$ to be convex, we need the following to be true: \newline 

        $\alpha m + (1 - \alpha)n \in C \cap L$ where $\alpha \in [0, 1]$ \newline 
        
        Since $C$ is convex, we know that 
        $\alpha m + (1 - \alpha)n \in C$ where $\alpha \in [0, 1]$ \newline


        Since $m$ and $n$ are in $C \cap L$, they must be points on $L$. We can define $L$ as $tm + (1 - t)n$ where $t \in \mathbb{R}$. It is easy to see that $\alpha m + (1 - \alpha)n \in L$ where $\alpha \in [0, 1]$ \newline 

        Since $\alpha m + (1 - \alpha)n \in L$ where $\alpha \in [0, 1]$ AND $\alpha m + (1 - \alpha)n \in C$ where $\alpha \in [0, 1]$, we can conclude by saying that: 

        $\alpha m + (1 - \alpha)n \in C \cap L$ where $\alpha \in [0, 1]$ \newline 

        Hence, $C \cap L$ is convex as well! 
        \end{itemize}

        
        Step 2: Show that a set is affine if and only if its intersection with any line is affine. 
        \begin{itemize}
            \item Part 1: The "If" Part \newline
        We know that the intersection of the set and any line is affine. Our goal is to show that the set itself is affine. Let us start with a set $C$. Let us also denote an arbitrary line as $L$. Let us denote two distinct points $m$ and $n$ that are in $C \cap L$. Since we know that $C \cap L$ is affine, we can state the following: \newline 

        $\alpha m + (1 - \alpha)n \in C \cap L$ where $\alpha \in \mathbb{R}$ \newline 

        This leads us to the following: \newline 

        $\alpha m + (1 - \alpha)n \in C$ where $\alpha \in \mathbb{R}$ \newline 
        $\alpha m + (1 - \alpha)n \in L$ where $\alpha \in \mathbb{R}$ \newline 
        
        Hence, by definition, $C$ is affine as well! 
        
            \item Part 2: The "Only If" Part \newline 
        We know that the set is affine. Our goal is to prove that the intersection of the set and any line is affine as well. Let us start with an affine set $C$. Let us also denote an arbitrary line as $L$. \newline 

        Let us denote two distinct points $m$ and $n$ that are in $C \cap L$. For $C \cap L$ to be affine, we need the following to be true: \newline 

        $\alpha m + (1 - \alpha)n \in C \cap L$ where $\alpha \in \mathbb{R}$ \newline 

        Since $C$ is affine, we know that \newline 
        $\alpha m + (1 - \alpha)n \in C$ where $\alpha \in \mathbb{R}$ \newline 

        Since $m$ and $n$ are in $C \cap L$, they must be points on $L$. We can define $L$ as $tm + (1 - t)n$ where $t \in \mathbb{R}$. It is easy to see that $\alpha m + (1 - \alpha)n \in L$ where $\alpha \in \mathbb{R}$ \newline 

        Since $\alpha m + (1 - \alpha)n \in L$ where $\alpha \in \mathbb{R}$ AND $\alpha m + (1 - \alpha)n \in C$ where $\alpha \in \mathbb{R}$, we can conclude by saying that: 

        $\alpha m + (1 - \alpha)n \in C \cap L$ where $\alpha \in \mathbb{R}$ \newline 

        Hence, $C \cap L$ is affine as well! 


        \end{itemize}
        
    \end{solution}
\end{problem}

\end{document}





