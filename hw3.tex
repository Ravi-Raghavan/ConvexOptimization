\documentclass{article}

\setlength{\headsep}{0.75 in}
\setlength{\parindent}{0 in}
\setlength{\parskip}{0.1 in}

%=====================================================
% Add PACKAGES Here (You typically would not need to):
%=====================================================

\usepackage{xcolor}
\usepackage[margin=1in]{geometry}
\usepackage{amsmath,amsthm}
\usepackage{fancyhdr}
\usepackage{enumitem}
\usepackage{graphicx}
\usepackage{amsmath, amssymb}  % Include the amsmath and amssymb packages for mathematical symbols

%=====================================================
% Ignore This Part (But Do NOT Delete It:)
%=====================================================

\theoremstyle{definition}
\newtheorem{problem}{Problem}
\newtheorem*{fun}{Fun with Algorithms}
\newtheorem*{challenge}{Challenge Yourself}
\def\fline{\rule{0.75\linewidth}{0.5pt}}
\newcommand{\finishline}{\begin{center}\fline\end{center}}
\newtheorem*{solution*}{Solution}
\newenvironment{solution}{\begin{solution*}}{{\finishline} \end{solution*}}
\newcommand{\grade}[1]{\hfill{\textbf{($\mathbf{#1}$ points)}}}
\newcommand{\thisdate}{February 22, 2024}
\newcommand{\thissemester}{\textbf{Rutgers: Spring 2024}}
\newcommand{\thiscourse}{ECE 509: Convex Optimization} 
\newcommand{\thishomework}{Number} 
\newcommand{\thisname}{Name} 
\newcommand{\thisextension}{Yes/No} 

\headheight 40pt              
\headsep 10pt
\renewcommand{\headrulewidth}{0pt}
\pagestyle{fancy}

\newcommand{\thisheading}{
   \noindent
   \begin{center}
   \framebox{
      \vbox{\vspace{2mm}
    \hbox to 6.28in { \textbf{\thiscourse \hfill \thissemester} }
       \vspace{4mm}
       \hbox to 6.28in { {\Large \hfill Homework \#\thishomework \hfill} }
       \vspace{2mm}
         \hbox to 6.28in { { \hfill \thisdate  \hfill} }
       \vspace{2mm}
       \hbox to 6.28in { \emph{Name: \thisname \hfill Extension: \thisextension}}
      \vspace{2mm}}
      }
   \end{center}
   \bigskip
}

%=====================================================
% Some useful MACROS (you can define your own in the same exact way also)
%=====================================================


\newcommand{\ceil}[1]{{\left\lceil{#1}\right\rceil}}
\newcommand{\floor}[1]{{\left\lfloor{#1}\right\rfloor}}
\newcommand{\prob}[1]{\Pr\paren{#1}}
\newcommand{\expect}[1]{\Exp\bracket{#1}}
\newcommand{\var}[1]{\textnormal{Var}\bracket{#1}}
\newcommand{\set}[1]{\ensuremath{\left\{ #1 \right\}}}
\newcommand{\poly}{\mbox{\rm poly}}


%=====================================================
% Fill Out This Part With Your Own Information:
%=====================================================


\renewcommand{\thishomework}{3} %Homework number
\renewcommand{\thisname}{Ravi Raghavan} % Enter your name here
\renewcommand{\thisextension}{No} % Pick only one of the two options accordingly

\begin{document}

\thisheading
\vspace{-0.75cm}


%=====================================================
% LaTeX Tip: You can erase this part from here.... 
%=====================================================		

\finishline

%=====================================================
% LaTeX Tip: ... to here
%=====================================================	


\bigskip

\begin{problem} Suppose $f: \mathbb{R} \rightarrow \mathbb{R}$ is convex, and $a, b \in dom f$ with $a < b$
    \begin{enumerate}
        \item[(a)] Show that 
                    \begin{equation}
                        \label{eq:example}
                            f(x) \leq \frac{b - x}{b - a} f(a) + \frac{x - a}{b - a} f(b)
                    \end{equation}
                    for all $x \in [a, b]$. 

                    \begin{solution}
                        Let us have $\theta \in \mathbb{R}$ where $0 \leq \theta \leq 1$. Since $x \in [a, b]$, we can define $x$ as $\theta a + (1 - \theta)b $ \newline 

                        Based on the definition of a convex function, we know that $f(\theta a + (1 - \theta)b) \leq \theta f(a) + (1 - \theta) f(b)$ \newline 

                        $f(x) \leq \theta f(a) + (1 - \theta) f(b)$ \newline 

                        $x = \theta a + b - \theta b$ \newline 

                        $x - b = \theta (a - b)$ \newline 

                        $\frac{x - b}{a - b} = \frac{b - x}{b - a} = \theta$ \newline 

                        $1 - \theta = 1 - \frac{b - x}{b - a} = \frac{x - a}{b - a}$ \newline 

                        $f(x) \leq \frac{b - x}{b - a} f(a) + \frac{x - a}{b - a} f(b)$ \newline 
                    \end{solution}
        \item[(b)] Show that 
                    \begin{equation}
                        \label{eq:example}
                            \frac{f(x) - f(a)}{x - a} \leq \frac{f(b) - f(a)}{b - a} \leq \frac{f(b) - f(x)}{b - x} 
                    \end{equation}
                    for all $x \in (a, b)$. Draw a sketch that illustrates this inequality

                    \begin{solution}
                        \begin{equation}
                        \label{eq:example}
                            f(x) \leq \frac{b - x}{b - a} f(a) + \frac{x - a}{b - a} f(b)
                    \end{equation}

                    Multiply both sides by $b - a$ \newline 

                    \begin{equation}
                        \label{eq:example}
                            f(x) (b - a) \leq (b - x) f(a) + (x - a) f(b)
                    \end{equation}

                    \begin{equation}
                        \label{eq:example}
                            f(x) (b - a) \leq (b - a) f(a) - (x - a) f(a) + (x - a) f(b)
                    \end{equation}

                    \begin{equation}
                        \label{eq:example}
                            (f(x) - f(a)) (b - a) \leq - (x - a) f(a) + (x - a) f(b)
                    \end{equation}

                    \begin{equation}
                        \label{eq:example}
                            (f(x) - f(a)) (b - a) \leq (x - a) (f(b) - f(a))
                    \end{equation}

                    Divide both sides by $x - a$

                    \begin{equation}
                        \label{eq:example}
                            \frac{(f(x) - f(a))}{x - a} (b - a) \leq (f(b) - f(a))
                    \end{equation}

                    Divide both sides by $b - a$

                    \begin{equation}
                        \label{eq:example}
                            \frac{(f(x) - f(a))}{x - a}  \leq \frac{(f(b) - f(a))}{b - a}
                    \end{equation}

Now, let's start with the same inequality, $f(x) (b - a) \leq (b - x) f(a) + (x - a) f(b)$, and work it in a new direction


                    \begin{equation}
                        \label{eq:example}
                            f(x) (b - a) \leq (b - x) f(a) + (x - a) f(b)
                    \end{equation}

                    \begin{equation}
                        \label{eq:example}
                            f(x) (b - a) \leq (b - x) f(a) + (b - a) f(b) - (b - x) f(b)
                    \end{equation}

                    \begin{equation}
                        \label{eq:example}
                            (f(x) - f(b)) (b - a) \leq (b - x) (f(a) - f(b))
                    \end{equation}

                    \begin{equation}
                        \label{eq:example}
                            (f(b) - f(x)) (b - a) \geq (b - x) (f(b) - f(a))
                    \end{equation}

Multiply each side by $\frac{1}{(b - x)(b - a)}$

                    \begin{equation}
                        \label{eq:example}
                            \frac{f(b) - f(x)}{b - x} \geq \frac{f(b) - f(a)}{b - a}
                    \end{equation}

Put together we get: \newline 

\begin{equation}
                        \label{eq:example}
                            \frac{f(x) - f(a)}{x - a}  \leq \frac{f(b) - f(a)}{b - a} \leq \frac{f(b) - f(x)}{b - x} 
                    \end{equation}

Note: The sketch is on the next page. 
\newpage 
                    \begin{figure}[h!]
        \centering
        \includegraphics[width=\textwidth]{1B.jpg}
        \caption{Sketch of Inequality in (B)}
    \end{figure}  

    Geometrically, this sketch depicts the inequality in Part (B). It shows that the slope of the line connecting points $(a, f(a))$ and $(b, f(b))$ is GREATER than the slope of the line connecting points $(a, f(a))$ and $(x, f(x))$. The sketch also shows that slope of the line connecting points $(a, f(a))$ and $(b, f(b))$ is LESS than the slope of the line connecting points $(x, f(x))$ and $(b, f(b))$
                    
                    \end{solution}
                    
        \item[(c)] Suppose $f$ is differentiable. Use the result in (b) to show that 
        \begin{equation}
                        \label{eq:example}
                            f'(a) \leq  \frac{f(b) - f(a)}{b - a} \leq f'(b)
                    \end{equation}

                \begin{solution}

                In Part (b) we proved that $\frac{f(x) - f(a)}{x - a}  \leq \frac{f(b) - f(a)}{b - a} \leq \frac{f(b) - f(x)}{b - x}$ \newline 

                Let's decompose this inequality into its two components: \newline 

                Component 1: $\frac{f(x) - f(a)}{x - a}  \leq \frac{f(b) - f(a)}{b - a}$ 

                Component 2: $\frac{f(b) - f(a)}{b - a} \leq \frac{f(b) - f(x)}{b - x}$

                If we take the limit as $x \rightarrow a$ for Component 1, we see that: \newline 

                $f'(a) \leq \frac{f(b) - f(a)}{b - a}$ \newline 

                If we take the limit as $x \rightarrow b$ for Component 2, we see that: \newline 

                $\frac{f(b) - f(a)}{b - a} \leq f'(b)$ \newline 

                Putting these results together, we get: $f'(a) \leq  \frac{f(b) - f(a)}{b - a} \leq f'(b)$

                We can also prove this result via a property of Convex Functions: \newline 
                We know that, for a convex function, the first-order Taylor approximation is in fact a global underestimator of the function. Hence, we can say the following: \newline 

                    $f(a) + f'(a) (x - a) \leq f(x)$ \newline 
                    $f'(a) \leq \frac{f(x) - f(a)}{x - a}$ \newline 

                    $f(b) + f'(b) (x - b) \leq f(x)$ \newline 
                    $f'(b) (b - x) \geq f(b) - f(x)$ \newline 
                    $f'(b) \geq \frac{f(b) - f(x)}{b - x}$ \newline
                    $\frac{f(b) - f(x)}{b - x} \leq f'(b)$ \newline


                    We know that $f'(a) \leq \frac{f(x) - f(a)}{x - a}$, $\frac{f(x) - f(a)}{x - a}  \leq \frac{f(b) - f(a)}{b - a} \leq \frac{f(b) - f(x)}{b - x}$, and $\frac{f(b) - f(x)}{b - x} \leq f'(b)$, we can say that: \newline 

                    $f'(a) \leq \frac{f(x) - f(a)}{x - a}  \leq \frac{f(b) - f(a)}{b - a} \leq \frac{f(b) - f(x)}{b - x} \leq f'(b)$ \newline 

                    $f'(a) \leq \frac{f(b) - f(a)}{b - a} \leq  f'(b)$ \newline 
                \end{solution}

        \item[(d)] Suppose $f$ is twice differentiable. Use the result in $(c)$ to show that $f''(a) \geq 0$ and $f''(b) \geq 0$

            \begin{solution}
                The result in Part (c) shows us that $f'(a) \leq  \frac{f(b) - f(a)}{b - a} \leq f'(b)$ \newline 

                From this, since $f'(a) \leq f'(b)$, we can go one step further and say that $\frac{f'(b) - f'(a)}{b - a} \geq 0$ \newline 

                From this, we can take the limit and see that: 

                $f''(a) = \lim_{b\to a} \frac{f'(b) - f'(a)}{b - a} \geq 0$ \newline 

                By extension, we can say that

                $f''(b) = \lim_{a\to b} \frac{f'(b) - f'(a)}{b - a} \geq 0$
                
                
                
            \end{solution}
    \end{enumerate}
\end{problem}

\begin{problem}
    Show that a continuous function $f: \mathbb{R}^n \rightarrow \mathbb{R}$ is convex if and only if for every line segment, its average value on the segment is less than or equal to the average of its values at the endpoints of the segment: For every $x, y \in \mathbb{R}^n$,

    \begin{equation}
        \label{eq:example}
            \int_{0}^{1} f(x + \lambda(y - x)) \,d \lambda \leq \frac{f(x) + f(y)}{2}
    \end{equation}

    \begin{solution}
        Step 1: The "Only If" Part \newline 
        We know that $f$ is convex and that $f$ is continuous. Our goal is to prove that, for every line segment, its average value on the segment is less than or equal to the average of its values at the endpoints of the segment. Let's re-arrange the insides of the integral to write its mathematical equivalent: 

        \begin{equation}
        \label{eq:example}
            \int_{0}^{1} f((1 - \lambda)x + \lambda y) \,d \lambda
    \end{equation}

Based on the definition of convexity, we know that $f((1 - \lambda)x + \lambda y) \leq \lambda f(y) + (1 - \lambda) f(x)$
    \begin{equation}
        \label{eq:example}
            \int_{0}^{1} f((1 - \lambda)x + \lambda y) \,d \lambda \leq \int_{0}^{1} \lambda f(y) + (1 - \lambda) f(x) \,d \lambda
    \end{equation}

Let's evaluate this integral \newline 
\begin{equation}
        \label{eq:example}
            \int_{0}^{1} \lambda f(y) + (1 - \lambda) f(x) \,d \lambda =  \int_{0}^{1} \lambda f(y) \,d \lambda +  \int_{0}^{1} (1 - \lambda) f(x) \,d \lambda
    \end{equation}

    \begin{equation}
        \label{eq:example}
            f(y) \int_{0}^{1} \lambda \,d \lambda +  f(x) \int_{0}^{1} (1 - \lambda) \,d \lambda
    \end{equation}

    \begin{equation}
        \label{eq:example}
            f(y) ([\frac{\lambda^2}{2}]^1_0) +  f(x) ([\lambda - \frac{\lambda^2}{2}]^1_0)
    \end{equation}

    \begin{equation}
        \label{eq:example}
            f(y) (\frac{1}{2}) +  f(x) (\frac{1}{2}) = \frac{f(x) + f(y)}{2}
    \end{equation}

We have shown that: 
    \begin{equation}
        \label{eq:example}
            \int_{0}^{1} f(x + \lambda(y - x)) \,d \lambda \leq \frac{f(x) + f(y)}{2}
    \end{equation}

Step 2: "If" Part \newline 
We know that $\int_{0}^{1} f(x + \lambda(y - x)) \,d \lambda \leq \frac{f(x) + f(y)}{2}$ and that $f$ is continuous. Our task is to prove that $f$ is convex. Let's approach this proof by proving the contrapositive. Let's say that $f$ is NOT convex. This means that there exists a $\lambda_c \in [0, 1]$, for some $x$ and $y$, where  $f((1 - \lambda_c)x + \lambda_c y) > \lambda_c f(y) + (1 - \lambda_c) f(x)$ 

Let's define $F(\lambda) = f((1 - \lambda)x + \lambda y) - \lambda f(y) - (1 - \lambda) f(x)$. Since $f$ is continuous, we know that $F$ is continuous. We also know that, at $\lambda_c$, $F$ will have a positive value. Let $\lambda_b$ be the first point before $\lambda_c$ where $F$ attains a value of 0. Let $\lambda_d$ be the first point after $\lambda_c$ where $F$ attains a value of 0. 

We can clearly see that, on the open interval $(\lambda_b, \lambda_d)$, $F(\lambda) > 0$. This means that for all $\lambda$ on the open interval $(\lambda_b, \lambda_d)$,  we can see that $f((1 - \lambda)x + \lambda y) > \lambda f(y) + (1 - \lambda) f(x)$

Let $j = (1 - \lambda_b)x + \lambda_b y$ and $k = (1 - \lambda_d)x + \lambda_d y$ \newline
$\int_{\lambda_b}^{\lambda_d} f((1 - \lambda)x + \lambda y) \,d \lambda > \int_{\lambda_b}^{\lambda_d} \lambda f(y) + (1 - \lambda) f(x) \,d \lambda$ \newline 
Hence, we can say that, if we make $\theta \in [0, 1]$,  $\int_{0}^{1} f(j + \theta(k - j)) \,d \theta > \int_{0}^{1} \theta f(k) + (1 - \theta) f(j) \,d \theta$

Since we know, from our previous work in Step 1, that $\int_{0}^{1} \theta f(k) + (1 - \theta) f(j) \,d \theta = \frac{f(j) + f(k)}{2}$

Hence, we can see that $\int_{0}^{1} f(j + \theta(k - j)) \,d \theta > \frac{f(j) + f(k)}{2}$

By proving the contrapositive, we have completed the proof
    \end{solution}
    
\end{problem}

\begin{problem}
    Suppose $f: \mathbb{R}^n \rightarrow \mathbb{R}$ is convex with $dom f = \mathbb{R}^n$, and bounded above on $\mathbb{R}^n$. Show that $f$ is constant

    \begin{solution}
        Let's do a proof by contradiction. We will first start with the assumption that $f$ is NOT constant. So there exist $\alpha, \beta$ on the domain of $f$ such that $f(\alpha) < f(\beta)$. Let $g(\theta) = f(\alpha + \theta(\beta - \alpha))$. We know that $f$ is convex. Furthermore, we learned that convex functions, when restricted to any line in its domain is convex as well. Hence, we can clearly see that $g$ is convex as well. We can also clearly see that $g(0) < g(1)$ \newline 

        Based on the definition of convex functions, we know that: \newline 
        $g(1) \leq (1 - \frac{1}{\theta}) g(0) + \frac{1}{\theta} g(\theta)$. 

        for values of $\theta$ that satisfy $\theta > 1$ \newline 

        Let's rearrange the inequality: \newline 

        $\frac{1}{\theta} g(\theta) \geq g(1) - (1 - \frac{1}{\theta}) g(0)$ \newline 
        $g(\theta) \geq \theta g(1) - (\theta - 1) g(0)$ \newline 
        $g(\theta) \geq g(0) + \theta (g(1) - g(0))$ \newline 


        Since $g(1) > g(0)$, we know that $g(1) - g(0)$ is positive. Hence, the term $g(0) + \theta (g(1) - g(0))$, as $\theta$ gets larger and larger, will become larger and larger. 

        Based on this, as $\theta \rightarrow \infty$, we can see that $g(\theta)$ grows without bound. This contradicts the notion that $f$ is bounded above on $\mathbb{R}^n$
        
        
    \end{solution}  
\end{problem}

\begin{problem}
\textit{Minimizing a quadratic function.} Consider the problem of minimizing a quadratic function: \newline 
minimize $f(x) = \frac{1}{2} x^TPx + q^Tx + r,$ \newline 
where $P \in S^n$ (but we do not assume $P \succcurlyeq 0$)

\begin{enumerate}
    \item[(a)] Show that if P $\not\succcurlyeq$ 0, i.e., the objective function f is not convex, then the problem is unbounded below.

    \begin{solution}
        We are given that $P \not\succcurlyeq 0$. This means that $P$ is NOT positive semidefinite. \newline 

        There exists a $z$ such that $z^TPz < 0$. Let's say we have a constant $\lambda \in \mathbb{R}^n$ and let's plug in $\lambda z$ into our equation. 

        $f(\lambda z) = \frac{1}{2} (\lambda z)^TP(\lambda z) + q^T(\lambda z) + r$
        
        $f(\lambda z) = \frac{\lambda^2}{2} z^TPz + \lambda q^Tz + r$ \newline

        Since the quadratic term dominates and  $z^TPz < 0$, we can see that as $\lambda$ gets larger, this function's value will get closer and closer to $-\infty$

        As $\lambda \rightarrow \infty$, we see that $f(\lambda z) \rightarrow -\infty$. This means that the functions is UNBOUNDED below 

    \end{solution}

    \item[(b)] Now suppose that P $\succcurlyeq$ 0 (so the objective function is convex), but the optimality condition $Px^* = -q$ does not have a solution. Show that the problem is unbounded below. \newline 

    Since the optimality condition $Px^* = -q$ does not have a solution, $q$ is NOT in the column space of $P$. Hence, $q$ can be expressed as $q = q_2 + q_3$ where $q_2$ is the projection of $q$ onto the column space of $P$ and $q_3$ is orthogonal to the column space. \newline 


    Since $q_3$ is orthogonal to the column space of $P$, we can say that $q_3^TPq_3 = 0$. Let's have $\lambda q_3$ for some constant $\lambda \in \mathbb{R}^n$. 

    $f(\lambda q_3) = \frac{1}{2} (\lambda q_3)^TP(\lambda q_3) + q^T(\lambda q_3) + r$ \newline 

    $f(\lambda q_3) = \frac{\lambda^2}{2} q_3^TPq_3 + \lambda q^Tq_3 + r$ \newline

    We know that $q_3^TPq_3 = 0$. Let's further simplify \newline 

    $f(\lambda q_3) = \lambda q^Tq_3 + r$ \newline

    We know that $q = q_3 + q_2$. Let's substitute this for $q$ \newline 


    $f(\lambda q_3) = \lambda (q_3 + q_2)^Tq_3 + r$ \newline

    $f(\lambda q_3) = \lambda q_3^Tq_3 + \lambda q_2^Tq_3 + r$ \newline

    We know that $q_2$ and $q_3$ are orthogonal. Hence, $q_2^Tq_3 = 0$ \newline 

    $f(\lambda q_3) = \lambda q_3^Tq_3  + r$ \newline

    This is unbounded below. As $\lambda \rightarrow -\infty$, we see that $f(\lambda q_3) \rightarrow -\infty$
    
\end{enumerate}

\end{problem}

\begin{problem}
    \textit{Minimizing a quadratic-over-linear fractional function.} Consider the problem of minimizing the function $f: \mathbb{R}^n \rightarrow \mathbb{R}$, defined as:

    \begin{equation}
                        \label{eq:example}
                            f(x) = \frac{\|\mathbf{Ax - b}\|^2_2}{c^Tx + d}, \enspace dom f = \{x | c^Tx + d > 0\}
                    \end{equation}
We assume $rank A = n$ and $b \notin R(A)$ 

\begin{enumerate}
    \item[(a)] Show that f is closed.
    \begin{solution}
        Based on the description, it is clear to see that the domain of $f$ is open. Hence, if we prove that for every sequence $x_i \in dom f$ such that $\lim_{i} x_i = x \in bd \enspace dom f$, we have $\lim_{i\to\infty} f(x_i) = \infty$, then we can show that $f$ is closed. \newline 

        Since $b \notin R(A)$, we can see that $\|\mathbf{Ax - b}\|^2_2$ is lower bounded by a positive real number. Hence, for any given sequence $H$ where $H$ is fully contained in the domain of $f$ and the limit point of $H$ is on the boundary of the domain of $f$, we can see that as $H$ approaches its limit point, $f(x)$ will approach infinity. \newline 

        Let's further analyze this. Clearly, the boundary of the domain is when $c^Tx + d = 0$. Since $\|\mathbf{Ax - b}\|^2_2$  is lower bounded by a positive real number, we can say that $\|\mathbf{Ax - b}\|^2_2 \geq k$ where $k$ is some positive real number. So, we can rewrite our function as  $f(x) = \frac{\|\mathbf{Ax - b}\|^2_2}{c^Tx + d} \geq \frac{k}{c^Tx + d}$. Obviously, as we approach the boundary of the domain, the denominator of $\frac{k}{c^Tx + d}$ approaches 0 and $\frac{k}{c^Tx + d}$ approaches infinity. Since $f(x) = \frac{\|\mathbf{Ax - b}\|^2_2}{c^Tx + d} \geq \frac{k}{c^Tx + d}$, $f(x)$ will approach infinity as well! 
    \end{solution}
    \item[(b)] Show that the minimizer $x^*$ of $f$ is given by 
    \begin{equation}
                        \label{eq:example}
                            x^* = x_1 + tx_2
                    \end{equation}
    where $x_1 = (A^TA)^{-1} A^Tb$, $x_2 = (A^TA)^{-1}c$, and $t \in \mathbb{R}$ can be calculated by solving a quadratic equation.

    \begin{solution}
    To find the value of $x^*$ that minimizes $f$, we know that $\nabla f(x^*) = 0$ \newline 

        $\nabla f(x^*) = \frac{2A^T (Ax^* - b)(c^Tx^* + d)}{(c^Tx^* + d) ^ 2} - \frac{\|\mathbf{Ax^* - b}\|^2_2 (c)}{(c^Tx^* + d) ^ 2}$ \newline 

        $\nabla f(x^*) = \frac{2A^T (Ax^* - b)}{(c^Tx^* + d)} - \frac{\|\mathbf{Ax^* - b}\|^2_2 (c)}{(c^Tx^* + d) ^ 2}$ \newline 


        Here is my proposed strategy for the rest of this proof: \newline 
        To show that $x^* = x_1 + tx_2$ is a valid minimizer, we will substitute the given values of $x_1$ and $x_2$ and show that there is a valid value of $t$ such that $\nabla f(x^*) = 0$ \newline 

        $\nabla f(x^*) = \frac{2(A^TA(x_1 + tx_2) - A^Tb)}{(c^T(x_1 + tx_2) + d)} - \frac{\|\mathbf{A(x_1 + tx_2) - b}\|^2_2 (c)}{(c^T(x_1 + tx_2) + d) ^ 2}$ \newline 

        $A^TAx_1 = A^Tb$. We can use that to our advantage and substitute it in our expression

        $\nabla f(x^*) = \frac{2(A^Tb + tA^TAx_2 - A^Tb)}{(c^T(x_1 + tx_2) + d)} - \frac{\|\mathbf{A(x_1 + tx_2) - b}\|^2_2 (c)}{(c^T(x_1 + tx_2) + d) ^ 2}$ \newline 

        $\nabla f(x^*) = \frac{2tA^TAx_2}{(c^T(x_1 + tx_2) + d)} - \frac{\|\mathbf{A(x_1 + tx_2) - b}\|^2_2 (c)}{(c^T(x_1 + tx_2) + d) ^ 2}$ \newline 

        $A^TAx_2 = c$. We can use that to our advantage and substitute it in our expression \newline 

        $\nabla f(x^*) = \frac{2tA^TAx_2}{(c^T(x_1 + tx_2) + d)} - \frac{\|\mathbf{A(x_1 + tx_2) - b}\|^2_2 (A^TAx_2)}{(c^T(x_1 + tx_2) + d) ^ 2}$ \newline 

        $\nabla f(x^*) = A^TAx_2 (\frac{2t}{(c^T(x_1 + tx_2) + d)} - \frac{\|\mathbf{A(x_1 + tx_2) - b}\|^2_2}{(c^T(x_1 + tx_2) + d) ^ 2})$ \newline 


        Now, we must show that, there is a valid value of $t$ that makes this gradient 0. \newline 

        
        Now, let's set this gradient to 0 \newline 

        $A^TAx_2 (\frac{2t}{(c^T(x_1 + tx_2) + d)} - \frac{\|\mathbf{A(x_1 + tx_2) - b}\|^2_2}{(c^T(x_1 + tx_2) + d) ^ 2}) = 0$ \newline 

        $\frac{2t}{(c^T(x_1 + tx_2) + d)} - \frac{\|\mathbf{A(x_1 + tx_2) - b}\|^2_2}{(c^T(x_1 + tx_2) + d) ^ 2} = 0$ \newline 

         $t = \frac{\|\mathbf{Ax_1 + Atx_2 - b}\|^2_2 }{2(c^Tx_1 + tc^Tx_2 + d)}$ \newline 


         $2t^2 c^Tx_2 + 2t(c^Tx_1 + d) = t^2 \|\mathbf{Ax_2}\|^2_2 + 2t(Ax_1 - b)^T Ax_2 +  \|\mathbf{Ax_1 - b}\|^2_2 $ \newline 

         We can make some key observations here that allow us to simplify this expression
         
         \begin{itemize}
             \item $||Ax_2||^2_2 = (Ax_2)^T(Ax_2) = x_2A^TAx_2 = (A^TAx_2)^T x_2 = c^T x_2$
             \item $2t(Ax_1 - b)^TAx_2 = 2t(x_1^TA^T - b^T)Ax_2 = 2t(x_1^TA^TAx_2 - b^TAx_2) = 2t((A^TAx_1)^Tx_2 - b^TAx_2) = 2t((A^Tb)^Tx_2 - b^TAx_2) = 2t(b^TAx_2 - b^TAx_2) = 0$
         \end{itemize}

         $2t^2 c^Tx_2 + 2t(c^Tx_1 + d) = t^2 c^T x_2 +  \|\mathbf{Ax_1 - b}\|^2_2 $ \newline 

         $t^2 c^Tx_2 + 2t(c^Tx_1 + d) - \|\mathbf{Ax_1 - b}\|^2_2  = 0 $ \newline 

         The roots of this equation are: \newline 

         $t = \frac{-(c^Tx_1 + d) \pm \sqrt{(c^Tx_1 + d)^2 + c^Tx_2 \|\mathbf{Ax_1 - b}\|^2_2}}{c^Tx_2}$

        As given by the domain restriction, we just need to solve for the value of $t$ such that: \newline 

        $c^T(x^*) + d > 0$ \newline 
        $c^T(x_1 + tx_2) + d > 0$ \newline 

        Given the equation for the roots, we can proceed from there: \newline 

        $t = \frac{-(c^Tx_1 + d) \pm \sqrt{(c^Tx_1 + d)^2 + c^Tx_2 \|\mathbf{Ax_1 - b}\|^2_2}}{c^Tx_2}$ \newline 

        Multiply both sides by $c^Tx_2$ \newline 

        $c^Tx_2 t = -(c^Tx_1 + d) \pm \sqrt{(c^Tx_1 + d)^2 + c^Tx_2 \|\mathbf{Ax_1 - b}\|^2_2}$ \newline 

        $(c^Tx_1 + d) + c^Tx_2 t = \pm \sqrt{(c^Tx_1 + d)^2 + c^Tx_2 \|\mathbf{Ax_1 - b}\|^2_2}$ \newline 

        $c^T(x_1 + tx_2) + d = \pm \sqrt{(c^Tx_1 + d)^2 + c^Tx_2 \|\mathbf{Ax_1 - b}\|^2_2}$

        As I showed earlier, $||Ax_2||^2_2 = (Ax_2)^T(Ax_2) = x_2A^TAx_2 = (A^TAx_2)^T x_2 = c^T x_2$. Hence, I can rewrite the square root as such: \newline 

        $c^T(x_1 + tx_2) + d = \pm \sqrt{(c^Tx_1 + d)^2 + ||Ax_2||^2_2 \|\mathbf{Ax_1 - b}\|^2_2}$

        The inside of the square root, clearly, must be $\geq 0$. Hence, the square root will evaluate to a real number! This is good news for us. 

        Since we want $c^T(x_1 + tx_2) + d > 0$, we can just say that we want  
        $c^T(x_1 + tx_2) + d =  \sqrt{(c^Tx_1 + d)^2 + ||Ax_2||^2_2 \|\mathbf{Ax_1 - b}\|^2_2}$ \newline 

        $c^T(x_1 + tx_2) + d =  \sqrt{(c^Tx_1 + d)^2 + c^Tx_2 \|\mathbf{Ax_1 - b}\|^2_2}$ \newline 


        This means that $t = \frac{-(c^Tx_1 + d) + \sqrt{(c^Tx_1 + d)^2 + c^Tx_2 \|\mathbf{Ax_1 - b}\|^2_2}}{c^Tx_2}$ is both guaranteed to be a valid real number that also satisfies our domain constraint! 

        This completes the proof. 
        
        
    \end{solution}
\end{enumerate}
\end{problem}

\end{document}





