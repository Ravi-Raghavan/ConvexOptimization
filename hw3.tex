\documentclass{article}

\setlength{\headsep}{0.75 in}
\setlength{\parindent}{0 in}
\setlength{\parskip}{0.1 in}

%=====================================================
% Add PACKAGES Here (You typically would not need to):
%=====================================================

\usepackage{xcolor}
\usepackage[margin=1in]{geometry}
\usepackage{amsmath,amsthm}
\usepackage{fancyhdr}
\usepackage{enumitem}
\usepackage{graphicx}
\usepackage{amsmath, amssymb}  % Include the amsmath and amssymb packages for mathematical symbols

%=====================================================
% Ignore This Part (But Do NOT Delete It:)
%=====================================================

\theoremstyle{definition}
\newtheorem{problem}{Problem}
\newtheorem*{fun}{Fun with Algorithms}
\newtheorem*{challenge}{Challenge Yourself}
\def\fline{\rule{0.75\linewidth}{0.5pt}}
\newcommand{\finishline}{\begin{center}\fline\end{center}}
\newtheorem*{solution*}{Solution}
\newenvironment{solution}{\begin{solution*}}{{\finishline} \end{solution*}}
\newcommand{\grade}[1]{\hfill{\textbf{($\mathbf{#1}$ points)}}}
\newcommand{\thisdate}{February 20, 2024}
\newcommand{\thissemester}{\textbf{Rutgers: Spring 2024}}
\newcommand{\thiscourse}{ECE 509: Convex Optimization} 
\newcommand{\thishomework}{Number} 
\newcommand{\thisname}{Name} 
\newcommand{\thisextension}{Yes/No} 

\headheight 40pt              
\headsep 10pt
\renewcommand{\headrulewidth}{0pt}
\pagestyle{fancy}

\newcommand{\thisheading}{
   \noindent
   \begin{center}
   \framebox{
      \vbox{\vspace{2mm}
    \hbox to 6.28in { \textbf{\thiscourse \hfill \thissemester} }
       \vspace{4mm}
       \hbox to 6.28in { {\Large \hfill Homework \#\thishomework \hfill} }
       \vspace{2mm}
         \hbox to 6.28in { { \hfill \thisdate  \hfill} }
       \vspace{2mm}
       \hbox to 6.28in { \emph{Name: \thisname \hfill Extension: \thisextension}}
      \vspace{2mm}}
      }
   \end{center}
   \bigskip
}

%=====================================================
% Some useful MACROS (you can define your own in the same exact way also)
%=====================================================


\newcommand{\ceil}[1]{{\left\lceil{#1}\right\rceil}}
\newcommand{\floor}[1]{{\left\lfloor{#1}\right\rfloor}}
\newcommand{\prob}[1]{\Pr\paren{#1}}
\newcommand{\expect}[1]{\Exp\bracket{#1}}
\newcommand{\var}[1]{\textnormal{Var}\bracket{#1}}
\newcommand{\set}[1]{\ensuremath{\left\{ #1 \right\}}}
\newcommand{\poly}{\mbox{\rm poly}}


%=====================================================
% Fill Out This Part With Your Own Information:
%=====================================================


\renewcommand{\thishomework}{3} %Homework number
\renewcommand{\thisname}{Ravi Raghavan} % Enter your name here
\renewcommand{\thisextension}{No} % Pick only one of the two options accordingly

\begin{document}

\thisheading
\vspace{-0.75cm}


%=====================================================
% LaTeX Tip: You can erase this part from here.... 
%=====================================================		

\finishline

%=====================================================
% LaTeX Tip: ... to here
%=====================================================	


\bigskip

\begin{problem} Suppose $f: \mathbb{R} \rightarrow \mathbb{R}$ is convex, and $a, b \in dom f$ with $a < b$
    \begin{enumerate}
        \item[(a)] Show that 
                    \begin{equation}
                        \label{eq:example}
                            f(x) \leq \frac{b - x}{b - a} f(a) + \frac{x - a}{b - a} f(b)
                    \end{equation}
                    for all $x \in [a, b]$. 

                    \begin{solution}
                        Let us have $\theta \in \mathbb{R}$ where $0 \leq \theta \leq 1$. Since $x \in [a, b]$, we can define $x$ as $\theta a + (1 - \theta)b $ \newline 

                        Based on the definition of a convex function, we know that $f(\theta a + (1 - \theta)b) \leq \theta f(a) + (1 - \theta) f(b)$ \newline 

                        $f(x) \leq \theta f(a) + (1 - \theta) f(b)$ \newline 

                        $x = \theta a + b - \theta b$ \newline 

                        $x - b = \theta (a - b)$ \newline 

                        $\frac{x - b}{a - b} = \frac{b - x}{b - a} = \theta$ \newline 

                        $1 - \theta = 1 - \frac{b - x}{b - a} = \frac{x - a}{b - a}$ \newline 

                        $f(x) \leq \frac{b - x}{b - a} f(a) + \frac{x - a}{b - a} f(b)$ \newline 
                    \end{solution}
        \item[(b)] Show that 
                    \begin{equation}
                        \label{eq:example}
                            \frac{f(x) - f(a)}{x - a} \leq \frac{f(b) - f(a)}{b - a} \leq \frac{f(b) - f(x)}{b - x} 
                    \end{equation}
                    for all $x \in (a, b)$. Draw a sketch that illustrates this inequality

                    \begin{solution}
                        \begin{equation}
                        \label{eq:example}
                            f(x) \leq \frac{b - x}{b - a} f(a) + \frac{x - a}{b - a} f(b)
                    \end{equation}

                    Multiply both sides by $b - a$ \newline 

                    \begin{equation}
                        \label{eq:example}
                            f(x) (b - a) \leq (b - x) f(a) + (x - a) f(b)
                    \end{equation}

                    \begin{equation}
                        \label{eq:example}
                            f(x) (b - a) \leq (b - a) f(a) - (x - a) f(a) + (x - a) f(b)
                    \end{equation}

                    \begin{equation}
                        \label{eq:example}
                            (f(x) - f(a)) (b - a) \leq - (x - a) f(a) + (x - a) f(b)
                    \end{equation}

                    \begin{equation}
                        \label{eq:example}
                            (f(x) - f(a)) (b - a) \leq (x - a) (f(b) - f(a))
                    \end{equation}

                    Divide both sides by $x - a$

                    \begin{equation}
                        \label{eq:example}
                            \frac{(f(x) - f(a))}{x - a} (b - a) \leq (f(b) - f(a))
                    \end{equation}

                    Divide both sides by $b - a$

                    \begin{equation}
                        \label{eq:example}
                            \frac{(f(x) - f(a))}{x - a}  \leq \frac{(f(b) - f(a))}{b - a}
                    \end{equation}


                    \begin{equation}
                        \label{eq:example}
                            f(x) (b - a) \leq (b - x) f(a) + (x - a) f(b)
                    \end{equation}

                    \begin{equation}
                        \label{eq:example}
                            f(x) (b - a) \leq (b - x) f(a) + (b - a) f(b) - (b - x) f(b)
                    \end{equation}

                    \begin{equation}
                        \label{eq:example}
                            (f(x) - f(b)) (b - a) \leq (b - x) (f(a) - f(b))
                    \end{equation}

                    \begin{equation}
                        \label{eq:example}
                            (f(b) - f(x)) (b - a) \geq (b - x) (f(b) - f(a))
                    \end{equation}

                    \begin{equation}
                        \label{eq:example}
                            \frac{f(b) - f(x)}{b - x} \geq \frac{f(b) - f(a)}{b - a}
                    \end{equation}

Put together we get: \newline 

\begin{equation}
                        \label{eq:example}
                            \frac{f(x) - f(a)}{x - a}  \leq \frac{f(b) - f(a)}{b - a} \leq \frac{f(b) - f(x)}{b - x} 
                    \end{equation}
                    
                    \end{solution}
        \item[(c)] Suppose $f$ is differentiable. Use the result in (b) to show that 
        \begin{equation}
                        \label{eq:example}
                            f'(a) \leq  \frac{f(b) - f(a)}{b - a} \leq f'(b)
                    \end{equation}

                \begin{solution}
                    We know that, for a convex function, the first-order Taylor approximation is in fact a global underestimator of the function. Hence, we can say the following: \newline 

                    $f(a) + f'(a) (x - a) \leq f(x)$ \newline 
                    $f'(a) \leq \frac{f(x) - f(a)}{x - a}$ \newline 

                    $f(b) + f'(b) (x - b) \leq f(x)$ \newline 
                    $f'(b) (b - x) \geq f(b) - f(x)$ \newline 
                    $f'(b) \geq \frac{f(b) - f(x)}{b - x}$ \newline
                    $\frac{f(b) - f(x)}{b - x} \leq f'(b)$ \newline


                    We know that $f'(a) \leq \frac{f(x) - f(a)}{x - a}$, $\frac{f(x) - f(a)}{x - a}  \leq \frac{f(b) - f(a)}{b - a} \leq \frac{f(b) - f(x)}{b - x}$, and $\frac{f(b) - f(x)}{b - x} \leq f'(b)$, we can say that: \newline 

                    $f'(a) \leq \frac{f(x) - f(a)}{x - a}  \leq \frac{f(b) - f(a)}{b - a} \leq \frac{f(b) - f(x)}{b - x} \leq f'(b)$ \newline 

                    $f'(a) \leq \frac{f(b) - f(a)}{b - a} \leq  f'(b)$ \newline 
                \end{solution}

        \item[(d)] Suppose $f$ is twice differentiable. Use the result in $(c)$ to show that $f''(a) \geq 0$ and $f''(b) \geq 0$

            \begin{solution}
                From $(c)$, we know that $f'(a) \leq \frac{f(b) - f(a)}{b - a}$ and $f'(b) \geq \frac{f(b) - f(a)}{b - a}$ and that $f'(a) \leq f'(b)$ \newline 

                $(c)$ tells us that given $(a, b)$ where $a \leq b$, $f'(a) \leq f'(b)$ 

                We can define the second derivative $f''(a)$ as such: \newline 

                $f''(a) = \lim_{h\to 0} \frac{f'(a + h) - f'(a)}{h}$

                Since the function is twice differentiable, this limit exists and we can see that when the limit is approached from the right side(i.e. when $h > 0$), both the numerator and denominator of the fraction are positive. Hence, the second derivative is positive. On the other hand, when the limit is approached from the left side(i.e. when $h < 0$), both the numerator and denominator of the fraction are negative. Hence, the limit has to be $\geq 0$ \newline 

                We can define the second derivative $f''(b)$ as such: \newline 

                $f''(b) = \lim_{h\to 0} \frac{f'(b + h) - f'(b)}{h}$

                Since the function is twice differentiable, this limit exists and we can see that when the limit is approached from the right side(i.e. when $h > 0$), both the numerator and denominator of the fraction are positive. Hence, the second derivative is positive. On the other hand, when the limit is approached from the left side(i.e. when $h < 0$), both the numerator and denominator of the fraction are negative. Hence, the limit has to be $\geq 0$


                This shows that $f''(a) \geq 0$ and $f''(b) \geq 0$
                
            \end{solution}
    \end{enumerate}
\end{problem}

\begin{problem}
    Show that a continuous function $f: \mathbb{R}^n \rightarrow \mathbb{R}$ is convex if and only if for every line segment, its average value on the segment is less than or equal to the average of its values at the endpoints of the segment: For every $x, y \in \mathbb{R}^n$,

    \begin{equation}
        \label{eq:example}
            \int_{0}^{1} f(x + \lambda(y - x)) \,d \lambda \leq \frac{f(x) + f(y)}{2}
    \end{equation}

    \begin{solution}
        Step 1: The "Only If" Part \newline 
        We know that $f$ is convex and that $f$ is continuous. Our goal is to prove that, for every line segment, its average value on the segment is less than or equal to the average of its values at the endpoints of the segment.  \newline 

        Let's re-arrange the insides of the integral to write its mathematical equivalent: \newline 

        \begin{equation}
        \label{eq:example}
            \int_{0}^{1} f((1 - \lambda)x + \lambda y) \,d \lambda
    \end{equation}

Based on the definition of convexity, we know that $f((1 - \lambda)x + \lambda y) \leq \lambda f(y) + (1 - \lambda) f(x)$
    \begin{equation}
        \label{eq:example}
            \int_{0}^{1} f((1 - \lambda)x + \lambda y) \,d \lambda \leq \int_{0}^{1} \lambda f(y) + (1 - \lambda) f(x) \,d \lambda
    \end{equation}

Let's evaluate this integral \newline 
\begin{equation}
        \label{eq:example}
            \int_{0}^{1} \lambda f(y) + (1 - \lambda) f(x) \,d \lambda =  \int_{0}^{1} \lambda f(y) \,d \lambda +  \int_{0}^{1} (1 - \lambda) f(x) \,d \lambda
    \end{equation}

    \begin{equation}
        \label{eq:example}
            f(y) \int_{0}^{1} \lambda \,d \lambda +  f(x) \int_{0}^{1} (1 - \lambda) \,d \lambda
    \end{equation}

    \begin{equation}
        \label{eq:example}
            f(y) ([\frac{\lambda^2}{2}]^1_0) +  f(x) ([\lambda - \frac{\lambda^2}{2}]^1_0)
    \end{equation}

    \begin{equation}
        \label{eq:example}
            f(y) (\frac{1}{2}) +  f(x) (\frac{1}{2}) = \frac{f(x) + f(y)}{2}
    \end{equation}

We have shown that: 
    \begin{equation}
        \label{eq:example}
            \int_{0}^{1} f(x + \lambda(y - x)) \,d \lambda \leq \frac{f(x) + f(y)}{2}
    \end{equation}

Step 2: "If" Part \newline 
We know that $\int_{0}^{1} f(x + \lambda(y - x)) \,d \lambda \leq \frac{f(x) + f(y)}{2}$ and that $f$ is continuous. Our task is to prove that $f$ is convex. \newline 

Let's approach this proof by proving the contrapositive. Let's say that $f$ is NOT convex. This means that there exists a $\lambda_c \in [0, 1]$ where  $f((1 - \lambda_c)x + \lambda_c y) > \lambda_c f(y) + (1 - \lambda_c) f(x)$ \newline 

Let's define $F(\lambda) = f((1 - \lambda)x + \lambda y) - \lambda f(y) - (1 - \lambda) f(x)$. Since $f$ is continuous, we know that $F$ is continuous. 

We also know that, at $\lambda_c$, $F$ will have a positive value. Let $\lambda_b$ be the first point before $\lambda_c$ where $F$ crosses 0. Let $\lambda_d$ be the first point after $\lambda_c$ where $F$ crosses 0. \newline 

On the open interval $(\lambda_b, \lambda_d)$, $F(\lambda) > 0$ \newline 

Let $j = (1 - \lambda_b)x + \lambda_b y$ and $k = (1 - \lambda_d)x + \lambda_d y$ \newline 

Hence, we can say that $\int_{0}^{1} f(j + \theta(k - j)) \,d \theta > \int_{0}^{1} \theta f(k) + (1 - \theta) f(j) \,d \theta = \frac{f(j) + f(k)}{2}$
    \end{solution}
    
\end{problem}

\begin{problem}
    Suppose $f: \mathbb{R}^n \rightarrow \mathbb{R}$ is convex with $dom f = \mathbb{R}^n$, and bounded above on $\mathbb{R}^n$. Show that $f$ is constant

    \begin{solution}
        Let's do a proof by contradiction. We will first start with the assumption that $f$ is NOT constant. So there exist $\alpha, \beta$ on the domain of $f$ such that $f(\alpha) < f(\beta)$ \newline 

        Let $g(\theta) = f(\alpha + \theta(\beta - \alpha))$. Since $f$ is convex, $g$ is convex as well. We can clearly see that $g(0) < g(1)$ \newline 

        Based on the definition of convex functions, we know that: \newline 
        $g(1) \leq (1 - \frac{1}{\theta}) g(0) + \frac{1}{\theta} g(\theta)$. 

        This happens whenever $\theta > 1$ \newline 

        Let's rearrange the inequality: \newline 

        $\frac{1}{\theta} g(\theta) \geq g(1) - (1 - \frac{1}{\theta}) g(0)$ \newline 
        $g(\theta) \geq \theta g(1) - (\theta - 1) g(0)$ \newline 
        $g(\theta) \geq g(0) + \theta (g(1) - g(0))$ \newline 


        Based on this, as $\theta \rightarrow \infty$, we can see that $g(\theta)$ grows without bound. This contradicts the notion that $f$ is bounded above on $\mathbb{R}^n$
        
        
    \end{solution}  
\end{problem}

\begin{problem}
\textit{Minimizing a quadratic function.} Consider the problem of minimizing a quadratic function: \newline 
minimize $f(x) = \frac{1}{2} x^TPx + q^Tx + r,$ \newline 
where $P \in S^n$ (but we do not assume $P \succcurlyeq 0$)

\begin{enumerate}
    \item[(a)] Show that if P $\not\succcurlyeq$ 0, i.e., the objective function f is not convex, then the problem is unbounded below.

    \begin{solution}
        We are given that $P \not\succcurlyeq 0$. This means that $P$ is NOT positive semidefinite. \newline 

        There exists a $z$ such that $z^TPz < 0$. Now, let's plug in $\lambda z$ into our equation. 

        $f(\lambda z) = \frac{1}{2} (\lambda z)^TP(\lambda z) + q^T(\lambda z) + r$
        $f(\lambda z) = \frac{\lambda^2}{2} z^TPz + \lambda q^Tz + r$ \newline 

        As $\lambda \rightarrow \infty$, we see that $f(\lambda z) \rightarrow -\infty$. This means that the functions is UNBOUNDED below 

    \end{solution}

    \item[(b)] Now suppose that P $\succcurlyeq$ 0 (so the objective function is convex), but the optimality condition $Px^* = -q$ does not have a solution. Show that the problem is unbounded below. \newline 

    Since the optimality condition $Px^* = -q$ does not have a solution, $q$ is NOT in the column space of $P$. Hence, $q$ can be expressed as $q = q_2 + q_3$ where $q_2$ is the projection of $q$ onto the column space of $P$ and $q_3$ is orthogonal to the column space. \newline 


    Since $q_3$ is orthogonal to the column space of $P$, we can say that $q_3^TPq_3 = 0$. Let's have $\lambda q_3$. 

    $f(\lambda q_3) = \frac{1}{2} (\lambda q_3)^TP(\lambda q_3) + q^T(\lambda q_3) + r$ \newline 

    $f(\lambda q_3) = \frac{\lambda^2}{2} q_3^TPq_3 + \lambda q^Tq_3 + r$ \newline

    $f(\lambda q_3) = \lambda q^Tq_3 + r$ \newline

    $f(\lambda q_3) = \lambda (q_3 + q_2)^Tq_3 + r$ \newline

    $f(\lambda q_3) = \lambda q_3^Tq_3 + \lambda q_2^Tq_3 + r$ \newline

    $f(\lambda q_3) = \lambda q_3^Tq_3  + r$ \newline

    This is unbounded below. 
    
\end{enumerate}

\end{problem}

\end{document}





