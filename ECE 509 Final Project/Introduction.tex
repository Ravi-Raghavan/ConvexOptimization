\subsection{Theory: Smooth Functions and Convexity}
When discussing convergence, we will first discuss the underlying theory of Smooth Functions and Convexity. 

\subsubsection{Differentiability}

\textbf{Definition 1} (Jacobian). Let $\mathcal{F}: \mathbb{R}^d \rightarrow \mathbb{R}^p$ be differentiable, and $x \in \mathbb{R}^d$. Then, we note $\mathcal{D}\mathcal{F}(x)$ the \textbf{Jacobian} of $\mathcal{F}$ at x, which is the matrix defined by its first partial derivatives: \newline 
\begin{equation}
    [\mathcal{D}\mathcal{F}(x)]_{ij} = \frac{\partial{f_i}}{\partial{x_j}}(x), \enspace for \enspace i = 1, \dots, p, j = 1, \dots, d
\end{equation}

\noindent \textbf{Remark 2} (Gradient). If $f: \mathbb{R}^d \rightarrow \mathbb{R}$ is differentiable, then $\mathcal{D}f(x) \in \mathbb{R}^{1 x d}$ is a row vector, whose transpose is called the \textbf{gradient} of $f$ at x: $\nabla f(x) = \mathcal{D} f(x)^T \in \mathbb{R}^{d x 1}$ \newline 

\noindent \textbf{Definition 3} (Hessian). Let $f: \mathbb{R}^d \rightarrow \mathbb{R}$ is twice differentiable, and $x \in \mathbb{R}^d$. Then we note $\nabla^2 f(x)$ the \textbf{Hessian} of f at x, which is the matrix defined by its second-order partial derivatives:
\begin{equation}
    [\nabla^2 f(x)]_{ij} = \frac{\partial^2{f}}{\partial{x_i}\partial{x_j}}(x), \enspace for \enspace i, j = 1, \dots, d
\end{equation}
Consequently $\nabla^2 f(x)$ is a $d \enspace x \enspace d$ matrix. 

\noindent \textbf{Definition 4} (Lipschitz). Let $\mathcal{F}: \mathbb{R}^d \rightarrow \mathbb{R}^p$, and $L > 0$. We say that F is L-\textbf{Lipschitz} if
\begin{equation}
    \forall x, y \in \mathbb{R}^d, \enspace ||\mathcal{F}(y) - \mathcal{F}(x)|| \leq L ||y - x||
\end{equation}

\subsubsection{Convexity}
\noindent \textbf{Definition 5} (Jensen's Inequality). A function $f: \mathbb{R}^n \rightarrow \mathbb{R}$ is \textit{convex} if \textbf{dom} $f$ is a convex set and if for all $x, y \in dom f$, and $\theta$ with $0 \leq \theta \leq 1$, we have
\begin{equation}
    f(\theta x + (1 - \theta)y) \leq \theta f(x) + (1 - \theta) f(y)
\end{equation}

\noindent \textbf{Definition 6} (First Order Condition of Convexity). A function $f: \mathbb{R}^n \rightarrow \mathbb{R}$ is \textit{convex} if and only if \textbf{dom} $f$ is a convex set and 
\begin{equation}
    f(y) \geq f(x) + \nabla f(x)^T (y - x)
\end{equation}
holds for all $x, y \in dom f$

\noindent \textbf{Definition 7} (Second Order Condition of Convexity). A function $f: \mathbb{R}^n \rightarrow \mathbb{R}$ is \textit{convex} if and only if \textbf{dom} $f$ is a convex set and its Hessian is positive semi-definite: for all $x \in dom f$, 
\begin{equation}
    \nabla^2 f(x) \succeq 0
\end{equation}

\subsubsection{Strong Convexity}
\noindent \textbf{Definition 5} (Jensen's Inequality for Strong Convexity). A function $f: \mathbb{R}^n \rightarrow \mathbb{R}$ is \textit{p-strongly convex} if \textbf{dom} $f$ is a convex set and if for all $x, y \in dom f$, and $\theta$ with $0 \leq \theta \leq 1$, we have
\begin{equation}
    f(\theta x + (1 - \theta) y) \leq \theta f(x) + (1 - \theta) f(y) - \frac{p}{2} \theta (1 - \theta) ||x - y||^2_2
\end{equation}

\noindent \textbf{Definition 6} (First Order Condition for Strong Convexity). A function $f: \mathbb{R}^n \rightarrow \mathbb{R}$ is \textit{p-strongly convex} if and only if \textbf{dom} $f$ is a convex set and 
\begin{equation}
    f(y) \geq f(x) + \nabla(f(x))^T (y - x) + \frac{p}{2} ||y - x||^2_2
\end{equation}
holds for all $x, y \in dom f$

\noindent \textbf{Definition 7} (Second Order Condition for Strong Convexity). A function $f: \mathbb{R}^n \rightarrow \mathbb{R}$ is \textit{p-strongly convex} if and only if \textbf{dom} $f$ is a convex set and 
\begin{equation}
    \nabla^2 f(x) \succeq pI
\end{equation}

\subsubsection{Smoothness}
\noindent \textbf{Definition 8} (L-Smooth Functions). A function $f: \mathbb{R}^n \rightarrow \mathbb{R}$ and $L > 0$ is L-Smooth if it is differentiable and if $\nabla f: \mathbb{R}^n \rightarrow \mathbb{R}^n$ is $L-$Lipschitz: 
\begin{equation}
    \forall x, y \in \mathbb{R}^n, \enspace ||\nabla f(y) - \nabla f(x)|| \leq L ||y - x||
\end{equation}

\noindent As a subsequent node, $L-$Smooth functions have a quadratic upper bound: 
\begin{equation}
    \forall x, y \in \mathbb{R}^n, f(y) \leq f(x) + \langle \nabla f(x), y - x \rangle + \frac{L}{2} ||y - x||^2
\end{equation}

\subsubsection{Smoothness and Convexity}
\noindent \textbf{Lemma:} If $f: \mathbb{R}^d \rightarrow \mathbb{R}$ is convex and $L-$smooth, then $\forall x, y \in \mathbb{R}^d$, we have that: 
\begin{equation}
    \frac{1}{2L} ||\nabla f(y) - \nabla f(x)||^2 \leq f(y) - f(x) - \langle \nabla f(x), y - x \rangle
\end{equation}

\noindent \textbf{Proof:} 
$f(x) - f(y) = f(x) - f(z) + f(z) - f(y)$ \newline 

\noindent We know that, due to the first order condition of convexity, $f(z) \geq f(x) + \nabla f(x)^T (z - x)$

\noindent $f(x) - f(z) \leq -\nabla f(x)^T (z - x) = \nabla f(x)^T(x - z)$ 

\noindent We also know that, due to the quadratic upper bound property of L-Smooth functions, \newline 
$f(z) \leq f(y) + \langle \nabla f(y), z - y \rangle + \frac{L}{2} ||z - y||^2$ \newline 
$f(z) - f(y) \leq \langle \nabla f(y), z - y \rangle + \frac{L}{2} ||z - y||^2$ \newline 

Hence, $f(x) - f(y) = f(x) - f(z) + f(z) - f(y) \leq \nabla f(x)^T(x - z) + \langle \nabla f(y), z - y \rangle + \frac{L}{2} ||z - y||^2$

\noindent Let's aim to minimize the right hand side. \newline 
Let's take the derivative of the right hand side with respect to $z$. This gradient is: \newline 
$-\nabla f(x) + \nabla f(y) + \frac{L}{2} (2z -2y)$ \newline 
$-\nabla f(x) + \nabla f(y) + L (z - y)$

Setting this to zero gives us: $z - y = \frac{1}{L}(\nabla f(x) - \nabla f(y))$ \newline 
$z = y - \frac{1}{L}(\nabla f(y) - \nabla f(x))$

Now that we have found the quantity to minimize the right hand side, we can substitute it in.

$\nabla f(x)^T(x - (y - \frac{1}{L}(\nabla f(y) - \nabla f(x)))) + \langle \nabla f(y), y - \frac{1}{L}(\nabla f(y) - \nabla f(x)) - y \rangle + \frac{L}{2} ||y - \frac{1}{L}(\nabla f(y) - \nabla f(x)) - y||^2$ \newline 

$\nabla f(x)^T(x - (y - \frac{1}{L}(\nabla f(y) - \nabla f(x)))) - \langle \nabla f(y),  \frac{1}{L}(\nabla f(y) - \nabla f(x))\rangle + \frac{L}{2} ||\frac{1}{L}(\nabla f(y) - \nabla f(x))||^2$ \newline 

$\nabla f(x)^T (x - y) + \frac{1}{L} \nabla f(x)^T (\nabla f(y) - \nabla f(x)) - \langle \nabla f(y),  \frac{1}{L}(\nabla f(y) - \nabla f(x))\rangle + \frac{1}{2L} ||\nabla f(y) - \nabla f(x)||^2$ \newline 

$\nabla f(x)^T (x - y) + \frac{1}{L} \langle \nabla f(x) - \nabla f(y), \nabla f(y) - \nabla f(x) \rangle + \frac{1}{2L} ||\nabla f(y) - \nabla f(x)||^2$ \newline 

$\nabla f(x)^T (x - y) - \frac{1}{L} \langle \nabla f(y) - \nabla f(x), \nabla f(y) - \nabla f(x) \rangle + \frac{1}{2L} ||\nabla f(y) - \nabla f(x)||^2$ \newline 

$\nabla f(x)^T (x - y) - \frac{1}{L} ||\nabla f(y) - \nabla f(x)||^2 + \frac{1}{2L} ||\nabla f(y) - \nabla f(x)||^2$ \newline 

$\nabla f(x)^T (x - y) - \frac{1}{2L} ||\nabla f(y) - \nabla f(x)||^2$ \newline 

We now see that $f(x) - f(y) \leq \nabla f(x)^T (x - y) - \frac{1}{2L} ||\nabla f(y) - \nabla f(x)||^2$

Rearranging terms will make us see that $\frac{1}{2L} ||\nabla f(y) - \nabla f(x)||^2 \leq f(y) - f(x) - \langle \nabla f(x), y - x \rangle$
\subsection{Gradient Descent}
\noindent \textbf{Gradient Descent Algorithm}. Let $x^{(0)} \in \mathbb{R}^d$, and let $\gamma > 0$ be a step size. The \textbf{Gradient Descent (GD)} algorithm defines a sequence $(x^{(t)})_{t \in \mathbb{N}}$ satisfying
\begin{equation}
    x^{(t + 1)}  = x^{(t)} - \gamma \nabla f(x^{(t)})
\end{equation}

\subsection{Function Definitions}
\noindent \textbf{Sum of Functions}. Let's say that we have a function $f: \mathbb{R}^d \rightarrow \mathbb{R}$ which can be expressed as: 
\begin{equation}
    f(x) = \frac{1}{n} \sum_{i=1}^{n} f_i(x)
\end{equation}
where $f_i: \mathbb{R}^d \rightarrow \mathbb{R}$. We can say that $f$ is a Sum of Functions

\noindent \textbf{Sum of Convex Functions}. Let's say that we have a function $f: \mathbb{R}^d \rightarrow \mathbb{R}$ which can be expressed as: 
\begin{equation}
    f(x) = \frac{1}{n} \sum_{i=1}^{n} f_i(x)
\end{equation}
where $f_i: \mathbb{R}^d \rightarrow \mathbb{R}$ and $f_i$ is convex. We can say that $f$ is a Sum of Convex Functions

\noindent \textbf{Sum of L-Smooth Functions}. Let's say that we have a function $f: \mathbb{R}^d \rightarrow \mathbb{R}$ which can be expressed as: 
\begin{equation}
    f(x) = \frac{1}{n} \sum_{i=1}^{n} f_i(x)
\end{equation}
where $f_i: \mathbb{R}^d \rightarrow \mathbb{R}$ and $f_i$ is $L_i$ smooth. We can say that $f$ is a Sum of L-Smooth Functions. Let $L_{max} = \max_{1, \dots, n} L_i$

\subsection{Stochastic Gradient Descent}
\noindent \textbf{Stochastic Gradient Descent Algorithm}. Let's say that we are minimizing a function $f: \mathbb{R}^d \rightarrow \mathbb{R}$ which is a sum of functions. It is assumed that $\arg \min f \neq \emptyset$ and that $f_i$ is unbounded below.  

\noindent Let $x^{(0)} \in \mathbb{R}^d$, and let $\gamma_t > 0$ be a sequence of step sizes. The \textbf{Stochastic Gradient Descent (GD)} algorithm defines a sequence $(x^{(t)})_{t \in \mathbb{N}}$ satisfying 
\begin{equation}
    i_t \in \{1, \dots, n\}
\end{equation}
\begin{equation}
    x^{(t + 1)}  = x^{(t)} - \gamma_t \nabla f_{i_t}(x^{(t)})
\end{equation}

\noindent Note $i_t$ is sampled with probability $\frac{1}{n}$

\noindent It is evident to see that the gradients used during Stochastic Gradient Descent is an unbiased estimator of $\nabla f(x)$. $E[\nabla f_i(x)] = \frac{1}{n} \sum_{i=1}^{n} \nabla f_i(x) = \nabla f(x)$

\subsection{Expected Smoothness and Variance}
\subsubsection{Expected Smoothness}
The goal of this section is to analyze the "expected properties" of $f_i$. \newline 
\noindent \textbf{Lemma}. Let's say that we have a function $f: \mathbb{R}^d \rightarrow \mathbb{R}$ which is a sum of Convex functions and a sum of L-Smooth functions, we can state the following: 
\begin{equation}
    \forall x, y \in \mathbb{R}^d, \enspace \frac{1}{2L_{max}} \mathbb{E}[||\nabla f_i(y) - \nabla f_i(x)||^2] \leq f(y) - f(x) - \langle \nabla f(x), y - x \rangle
\end{equation}

\noindent \textbf{Proof}: 
We can use the Lemma that we derived in Section 1.1.5 as well as the fact that $L_i \leq L_{max}$ to see the following: \newline 

$\frac{1}{2L_{max}} ||\nabla f_i(y) - \nabla f_i(x)||^2 \leq \frac{1}{2L_i} ||\nabla f_i(y) - \nabla f_i(x)||^2 \leq f_i(y) - f_i(x) - \langle \nabla f_i(x), y - x \rangle$ \newline 

$\frac{1}{2L_{max}} ||\nabla f_i(y) - \nabla f_i(x)||^2 \leq f_i(y) - f_i(x) - \langle \nabla f_i(x), y - x \rangle$ \newline 

$\frac{1}{n} (\frac{1}{2L_{max}} ||\nabla f_i(y) - \nabla f_i(x)||^2) \leq \frac{1}{n} (f_i(y) - f_i(x) - \langle \nabla f_i(x), y - x \rangle)$ \newline 

We are now prepared to take expectation. The above inequality can be duplicated for each value of $i$ and we can sum these inequalities across all values of $i$. Hence, we see the following: \newline 

$\frac{1}{2L_{max}} \mathbb{E}[||\nabla f_i(y) - \nabla f_i(x)||^2] \leq f(y) - f(x) - \langle \nabla f(x), y - x \rangle$

If we set $x = x^*$ and $y = x$, we can see that 

$\frac{1}{2L_{max}} \mathbb{E}[||\nabla f_i(x) - \nabla f_i(x^*)||^2] \leq f(x) - f(x^*) - \langle \nabla f(x^*), x - x^* \rangle$ \newline 

$\frac{1}{2L_{max}} \mathbb{E}[||\nabla f_i(x) - \nabla f_i(x^*)||^2] \leq f(x) - inf f$


\subsubsection{Variance}
It is rather intuitive that, given a particular value of $x$, if the values of $f_i$ at this value of $x$ exhibit a lower variance, this would indicate that our stochastic algorithm would converge faster. This section is going to dive deep into this variance concept and provide a few measurable quantities that will quantify this concept. 

\noindent \textbf{Definition}. Let's say we have a function that is a Sum of Functions. \textbf{Interpolation} holds if there exists a common $x^* \in \mathbb{R}^d$ such that $f_i(x^*) = inf f_i, \forall i = 1, \dots n$. We will state that interpolation occurs at $x^*$

\noindent \textbf{Lemma.} Let's say we have a function that is a Sum of Functions. If interpolation holds at $x^* \in \mathbb{R}^d$, then $x^* \in \arg \min f$

\noindent \textbf{Proof.} 
Since we know that interpolation holds at $x^*$, we know that $f_i(x^*) = inf f_i, \forall i = 1, \dots n$

$f(x^*) = \frac{1}{n} \sum_{i=1}^{n} f_i(x^*) = \frac{1}{n} \sum_{i=1}^{n} inf f_i \leq \frac{1}{n} \sum_{i=1}^{n} f_i(x) = f(x)$

\noindent \textbf{Definition.} Again, we are looking at functions that are Sum of Functions. Let us define a new quantity called \textbf{function noise}. 
\begin{equation}
    \Delta^*_f = inf f - \frac{1}{n} \sum_{i=1}^{n} inf f_i
\end{equation}

\noindent \textbf{Lemma.} Given the previous definition, we can state the following 
\begin{equation}
    \Delta^*_f \geq 0
\end{equation}

Interpolation Holds if and only if $\Delta^*_f = 0$ 

\noindent \textbf{Proof.} $\Delta^*_f = f(x^*) - \frac{1}{n} \sum_{i=1}^{n} inf f_i \geq f(x^*) - \frac{1}{n} \sum_{i=1}^{n} f_i(x^*) = f(x^*) - f(x^*) = 0$

Let's now prove that Interpolation Holds if and only if $\Delta^*_f = 0$. 

First we will prove the first direction. So let it be the case that interpolation holds. 

When interpolation holds, $inf f_i = f_i(x^*)$. Hence, we can say that $\Delta^*_f = f(x^*) - \frac{1}{n} \sum_{i=1}^{n} inf f_i = f(x^*) - \frac{1}{n} \sum_{i=1}^{n} f_i(x^*) = f(x^*) - f(x^*) = 0$

Now let's prove the second direction. Let it be the case that $\Delta^*_f = 0$. 

We know that $\Delta^*_f = f(x^*) - \frac{1}{n} \sum_{i=1}^{n} inf f_i \geq f(x^*) - \frac{1}{n} \sum_{i=1}^{n} f_i(x^*) = f(x^*) - f(x^*) = 0$ \newline 

This means that $\sum_{i=1}^{n} inf f_i = \sum_{i=1}^{n} f_i(x^*)$ \newline 

$\sum_{i=1}^{n} f_i(x^*) - inf f_i = 0$ \newline 

We know that $f_i(x^*) \geq inf f_i$. Hence, for the above equality to hold true, we must observe that $f_i(x^*) = inf f_i$ which means that Interpolation must hold

\noindent \textbf{Definition} Let us now work with a function that is a Sum of L-Smooth Functions. We will define a term, called \textbf{gradient noise} that is set up as follows: 

\begin{equation}
    \sigma_f^* = \inf_{x^* \in \arg \min f} V[\nabla f_i(x^*)]
\end{equation}

where $V[X] = E[|| X - E[X] ||^2]$ \newline 

\noindent \textbf{Lemma.} Let us say we have a function that is a sum of L-Smooth Functions. 

If we also have that this function is a sum of Convex functions, we can claim that 
\begin{equation}
    \sigma_f^* = V[\nabla f_i(x^*)], \forall x^* \in \arg \min f
\end{equation}

\noindent \textbf{Proof.} Let us denote $x_1, x^* \in \arg \min f$. If we can show that $V[\nabla f_i(x_1)] = V[\nabla f_i(x^*)]$, then we will have completed the proof 

$\frac{1}{2L_{max}} \mathbb{E}[||\nabla f_i(x_1) - \nabla f_i(x^*)||^2] \leq f(x_1) - f(x^*) - \langle \nabla f(x^*), x_1 - x^* \rangle$

We know that $f(x^*) = inf f$ and that $\nabla f(x^*) = 0$ \newline 

$\frac{1}{2L_{max}} \mathbb{E}[||\nabla f_i(x_1) - \nabla f_i(x^*)||^2] \leq f(x_1) - inf f = 0$

Since $||\nabla f_i(x_1) - \nabla f_i(x^*)||^2$ is a positive quantity, this must mean that $\mathbb{E}[||\nabla f_i(x_1) - \nabla f_i(x^*)||^2] = 0$ and that $||\nabla f_i(x_1) - \nabla f_i(x^*)|| = 0$ for all $i$. 

Hence, we have shown that $f_i(x_1) = f_i(x^*)$ and that, subsequently, $V[\nabla f_i(x_1)] = V[\nabla f_i(x^*)]$

\noindent From the aforementioned analysis it is clear that both the function noise and gradient noise measure how close/far away from interpolation we are. 

\noindent \textbf{Lemma}. Let us say that we have a function that is a Sum of L-Smooth Functions. 

\begin{enumerate}
    \item $\sigma_f^* \leq 2L_{max} \Delta^*_f$
    \item If each $f_i$ is $p$ strongly convex, then $2p\Delta^*_f \leq \sigma_f^*$
\end{enumerate}

\noindent \textbf{Lemma}.  Let us say that we have a function that is a Sum of L-Smooth Functions. 
\begin{equation}
    \forall x \in \mathbb{R}^d, \enspace \mathbb{E}[||\nabla f_i(x)||^2] \leq 2L_{max} (f(x) - \inf f) + 2L_{max} \Delta^*_f
\end{equation}

\noindent \textbf{Lemma. } Let us say that we have a function that is a Sum of L-Smooth and Convex Functions. 

\begin{equation}
\forall x \in \mathbb{R}^d, \enspace \mathbb{E}[||\nabla f_i(x)||^2] \leq 4L_{max} (f(x) - \inf f) + 2\sigma^*_f
\end{equation}

For the rest of this proof, let's pick an $x^* \in \arg \min f$

In our journey to prove this lemma, the first thing we can do is express $||\nabla f_i(x)||^2$ as follows: \newline 

$||\nabla f_i(x)||^2 = ||\nabla f_i(x) - \nabla f_i(x^*) + f_i(x^*)||^2$ \newline 

Based on the triangular inequality for norms, we know that $||x + y|| \leq ||x|| + ||y||$ \newline 

$||\nabla f_i(x) - \nabla f_i(x^*) + f_i(x^*)||^2 \leq ||\nabla f_i(x) - \nabla f_i(x^*)||^2 + 2 ||\nabla f_i(x) - \nabla f_i(x^*)|| ||\nabla f_i(x^*)|| + ||\nabla f_i(x^*)||^2$ \newline 

We know that $||\nabla f_i(x) - \nabla f_i(x^*) - f_i(x^*)||^2 \geq 0$ \newline 

$||\nabla f_i(x) - \nabla f_i(x^*)||^2 - 2 ||\nabla f_i(x) - \nabla f_i(x^*)|| ||\nabla f_i(x^*)|| + ||\nabla f_i(x^*)||^2 \geq 0$ \newline 

$||\nabla f_i(x) - \nabla f_i(x^*)||^2  + ||\nabla f_i(x^*)||^2 \geq 2 ||\nabla f_i(x) - \nabla f_i(x^*)|| ||\nabla f_i(x^*)||$ \newline 

We can express $||\nabla f_i(x) - \nabla f_i(x^*) + f_i(x^*)||^2 \leq ||\nabla f_i(x) - \nabla f_i(x^*)||^2 + 2 ||\nabla f_i(x) - \nabla f_i(x^*)|| ||\nabla f_i(x^*)|| + ||\nabla f_i(x^*)||^2$  as follows: \newline 

$||\nabla f_i(x) - \nabla f_i(x^*) + f_i(x^*)||^2 \leq 2||\nabla f_i(x) - \nabla f_i(x^*)||^2 + 2||\nabla f_i(x^*)||^2$ \newline 

$||\nabla f_i(x)||^2 \leq 2||\nabla f_i(x) - \nabla f_i(x^*)||^2 + 2||\nabla f_i(x^*)||^2$ \newline 

The next thing we can do is take the expectation over this inequality \newline

\noindent $\mathbb{E}[||\nabla f_i(x)||^2] \leq \mathbb{E} [2||\nabla f_i(x) - \nabla f_i(x^*)||^2 + 2||\nabla f_i(x^*)||^2]$ \newline 

$\mathbb{E}[||\nabla f_i(x)||^2] \leq 2\mathbb{E} [||\nabla f_i(x) - \nabla f_i(x^*)||^2 + ||\nabla f_i(x^*)||^2]$ \newline 

Earlier, we showed that $\frac{1}{2L_{max}} \mathbb{E}[||\nabla f_i(x) - \nabla f_i(x^*)||^2] \leq f(x) - inf f$ \newline 

This basically means that $\mathbb{E}[||\nabla f_i(x) - \nabla f_i(x^*)||^2] \leq 2L_{max} (f(x) - \inf f)$ \newline 

$2\mathbb{E}[||\nabla f_i(x) - \nabla f_i(x^*)||^2] \leq 4L_{max} (f(x) - \inf f)$

Since we have a function that is a Sum of L-Smooth and Convex Functions, we know that \newline 
\noindent $\sigma_f^* = V[\nabla f_i(x^*)], \forall x^* \in \arg \min f.$ \newline 

\noindent $\sigma_f^* = E[||\nabla f_i(x^*) - E[\nabla f_i(x^*)]||^2], \forall x^* \in \arg \min f.$ \newline 

We know that $E[\nabla f_i(x^*)] = \nabla f(x^*) = 0$ since $f$ is a Sum of Convex Functions. \newline 

In this case, 
\noindent $\sigma_f^* = E[||\nabla f_i(x^*)||^2]$ \newline 

$2 \sigma_f^* = 2 E[||\nabla f_i(x^*)||^2]$ \newline 

According to Linearity of Expectation, we know that \newline 
$2\mathbb{E} [||\nabla f_i(x) - \nabla f_i(x^*)||^2 + ||\nabla f_i(x^*)||^2] = 2\mathbb{E} [||\nabla f_i(x) - \nabla f_i(x^*)||^2] + 2 \mathbb{E}  [||\nabla f_i(x^*)||^2]$


Using this fact, we can further simplify this inequality: \noindent $\mathbb{E}[||\nabla f_i(x)||^2] \leq \mathbb{E} [2||\nabla f_i(x) - \nabla f_i(x^*)||^2 + 2||\nabla f_i(x^*)||^2]$ \newline 

\noindent $\mathbb{E}[||\nabla f_i(x)||^2] \leq 2\mathbb{E} [||\nabla f_i(x) - \nabla f_i(x^*)||^2] + 2 \mathbb{E}  [||\nabla f_i(x^*)||^2]$ \newline 

Substituting the values that we derived earlier(i.e. $2\mathbb{E}[||\nabla f_i(x) - \nabla f_i(x^*)||^2] \leq 4L_{max} (f(x) - \inf f)$ and $2 \sigma_f^* = 2 E[||\nabla f_i(x^*)||^2]$ \newline 

\noindent $\mathbb{E}[||\nabla f_i(x)||^2] \leq 4L_{max} (f(x) - \inf f) + 2 \sigma_f^*$